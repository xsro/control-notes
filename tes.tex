\documentclass[tikz,border=10pt]{standalone}
\usepackage{amsmath,amssymb} % 用于数学符号和公式
\begin{document}
\begin{tikzpicture}[
    >=stealth, % 箭头样式
    axis/.style={thick,->}, % 坐标轴样式
    func/.style={thick,red}, % 函数图像样式
    label/.style={font=\small}, % 标签字体样式
    fill区域/.style={red!20} % 积分区域填充色
]

% 1. 定义参数n(可修改n的值观察函数变化,此处以n=3为例)
\def\n{3}
\def\interval{1/(\n*\n)} % 区间长度1/n²,计算得1/9≈0.111

% 2. 绘制坐标轴
% x轴(时间t):范围0到1.2(留出足够空间显示标注),高度设为0
\draw[axis] (0,0) -- (1.2,0) node[right,label] {$t$};
% y轴(函数值f_n(t)):范围0到\n+1(高于最大值n,避免标注拥挤)
\draw[axis] (0,0) -- (0,\n+1) node[above,label] {$f_n(t)$};

% 3. 标注坐标轴关键刻度
% x轴刻度:0、1/n²、1(1为参考点,体现区间[0,1/n²]的“窄”)
\draw (0,2pt) -- (0,-2pt) node[below,label] {$0$};
\draw (\interval,2pt) -- (\interval,-2pt) node[below,label] {$\frac{1}{n^2}$};
\draw (1,2pt) -- (1,-2pt) node[below,label] {$1$};

% y轴刻度:0、n(函数最大值)
\draw (2pt,0) -- (-2pt,0) node[left,label] {$0$};
\draw (2pt,\n) -- (-2pt,\n) node[left,label] {$n$};

% 4. 绘制函数f_n(t)(分段函数:[0,1/n²]上为n,其余为0)
% 左半段:t∈[0,1/n²],f_n(t)=n(水平线段)
\draw[func] (0,\n) -- (\interval,\n);
% 右半段:t>1/n²,f_n(t)=0(沿x轴延伸)
\draw[func] (\interval,0) -- (1.1,0);
% 垂直连接段:t=1/n²处,从(1/n²,0)到(1/n²,n)(体现函数在该点的跳跃)
\draw[func,dashed] (\interval,0) -- (\interval,\n); % 虚线表示“跳跃点不包含右侧”
% 左端点标记:(0,n)为实心点(包含t=0)
\fill[func] (0,\n) circle (1.5pt);
% 右端点标记:(1/n²,n)为空心点(不包含t=1/n²),(1/n²,0)为实心点(包含t=1/n²)
\fill[white,draw=red] (\interval,\n) circle (1.5pt);
\fill[func] (\interval,0) circle (1.5pt);

% 5. 填充积分区域(t∈[0,1/n²],f_n(t)下方与x轴之间的矩形)
\fill[fill区域] (0,0) rectangle (\interval,\n);

% 6. 添加函数表达式及特性说明
% 函数表达式(右上角)
\node[above right,label,align=left] at (0.2,\n) {
    $f_n(t) = n \cdot \chi_{[0, \frac{1}{n^2}]}(t)$ \\
    ($\chi$为指示函数)
};

% 积分特性(右下角)
\node[below right,label,align=left] at (0.5,0.5) {
    积分收敛:$\int_0^\infty f_n(t)dt = \frac{1}{n} \to 0$ \\
    函数发散:$\max_{t \in \mathbb{R}} f_n(t) = n \to \infty$
};

% 7. 标注积分区域(左下角)
\node[below left,label] at (\interval/2,\n/2) {
    积分区域:面积$=n \cdot \frac{1}{n^2} = \frac{1}{n}$
};

\end{tikzpicture}
\end{document}