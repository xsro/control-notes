\chapter{非光滑控制系统:右端有界}

对于右端不连续的系统,目前仍然是学界的研究重点,主流有两种方法来刻画(\cite{poznyakVadimUtkinSliding2023})。
一个是以\cite{filippovDifferentialEquationsDiscontinuous1988}为代表的Filippov方法,一个是Utkin的equivalent control (EC)方法。
其余方法可以见短文\cite{utkinBriefCommentsDoubts2022}、杂志\cite{cortesDiscontinuousDynamicalSystems2008}等文献。

\section{不一定连续可微的解}



考虑如下的动态系统 \[
\dot{x}(t)=\mathcal{X}(x(t))\ x(t_0)=x_0
\tag{7}
\] 其中\(x\in \mathbb{R}^d\), \(d\)为一个正整数,
并且\(\mathcal{X}:\mathbb{R}^d \to \mathbb{R}^d\) \textbf{不需要连续}。
我们称\textbf{连续可微的解\(t \mapsto x(t)\)为经典(classical)解}。
显然,如果\(\mathcal{X}\)连续,那么方程所有解都是classical的。
不失一般性,我们认为\(t_0=0\),并且只考虑\(t>0\)的情况。



\textbf{Caratheodory解}是classical解的一般化。
粗略地说,Caratheodory解是满足微分方程(7)的Lebesgue积分形式(8)的绝对连续曲线
\[
x(t)=x(t_0)+\int_{t_0}^t X(x(s)) ds,\quad t>t_0
\tag{8}
\]
通过使用积分形式(8),Caratheodory解不再要求方程解必须所有时间都沿着向量场的方向。
也就是说,微分方程(7)need bot be satisfied on a set of measure zero.

\textbf{Filippov解}使用微分包含式(differential
inclusion)来替换微分方程(7)右侧 \[
\dot{x}(t)\in \mathcal{F} (x(t))
\] 其中\(\mathcal{F}:\mathbb{F}^d \to \mathfrak{B}(\mathbb{F}^d)\),
\(\mathfrak{B}(\mathbb{R}^d)\)为d维实数空间\(\mathbb{R}^d\)的所有子集的集合。
Filippov解是绝对连续曲线。
对于任意给定的状态\(x\),Filippov解不只关注向量场在\(x\)处的值,
Filippov解的思想是引入由向量场中\(x\)的领域决定的一组\textbf{方向集合}。
文献中常常使用集值映射(set-value map),
也就是说这种映射的值是一个集合,而不像标准的函数(映射)的值只有一个。

\begin{quote}
An ordinary differential inclusion says the derivative must lie in a
specified set, which may also depend on the function and independent
variable.
\end{quote}

Caratheodory解和Filippov解都不能完全解决非连续动态系统的问题,
围绕存在的问题由\textbf{Sample-and-hold}解以及其他的一些描述方法。

\subsection{解的存在性和唯一性}\label{ux89e3ux7684ux5b58ux5728ux6027ux548cux552fux4e00ux6027}

对于常微分方程而言,向量场如果只连续不能保证解的唯一性。
我们说解的一个性质弱,表示不是所有解满足这一性质。
我们说解的一个性质强,表示所有解满足这一性质。
所以设计控制器的思路可以是

\begin{enumerate}
\def\labelenumi{\arabic{enumi}.}

\item
  设计控制器并考虑控制器下的闭环系统
\item
  用一个集值映射将每一个状态映射到允许的输入产生的所有向量的集合,并将这个映射与控制系统关联起来,(原文表述如下)
\end{enumerate}

\begin{quote}
associate with the control system the set-valued map that assigns each
state to the set of all vectors generated by the allowable inputs and
consider the resulting differential inclusion.
\end{quote}

\subsection{classical
解的存在性}\label{classical-ux89e3ux7684ux5b58ux5728ux6027}

考虑微分方程: \[
\dot{x}(t)=X(x(t)),\quad x(0)=x_0
\tag{10}
\] 其中\(X:\mathbb{R}^d \to \mathbb{R}^d\)是一个向量场。
如果\(0=X(x_e)\),那么点\(x_e\in \mathbb{R}^d\)是(10)的一个平衡点。
在\([0,t_1]\)上一个(10)的classical解是一个连续可微的映射\(x:[0,t_1]\to\mathbb{R}^d\)。
不失一般性,我们只考虑从时间\(t_0=0\)开始的解。 如果解\(t\mapsto x(t)\)
不能在时间上延申(extend),
也就是说解不是定义域内更大的一个时间区间上的解的截断,
那么称该解为最大解(maximal solution)。
最大解的定义暗示了解的区间只能是\([0,T),T>0\)或\([0,\infty)\)。

Peano's theorem 说明了连续的向量场可以保证classical解存在:

\begin{quote}
(Proposition 1) 令\(X:\mathbb{R}^d\to \mathbb{R}^d\)是连续向量场。
于是,对于所有\(x_0\in\mathbb{R}^d\),
\textbf{存在}一个(10)的classical解,该解满足\(x(0)=x_0\)
\end{quote}

\subsection{classical
解的唯一性}\label{classical-ux89e3ux7684ux552fux4e00ux6027}

\begin{quote}
(Proposition 2) 令\(x:\mathbb{R}^d\to\mathbb{R}^d\)连续,
假设对于所有的\(x\in\mathbb{R}^d\),
存在一个\(\epsilon>0\)使得\(X\)是在状态\(x\)的\(\epsilon\)领域\(B(x,\epsilon)\)上单侧Lipschitz连续。
然后,对于所有\(x_0\in \mathbb{R}^d\),存在一个起始于\(x(0)=x_0\)的(10)的\textbf{唯一的}classical
解
\end{quote}

\subsection{classical
解存在性和唯一性示例}\label{classical-ux89e3ux5b58ux5728ux6027ux548cux552fux4e00ux6027ux793aux4f8b}

下面的例子说明如果向量场不连续,那么(10)可能不存在经典解

\begin{quote}
考虑如下向量场:\(X:\mathbb{R}\to\mathbb{R}\) \[
X(x)=\begin{cases}
-1, & x>0\\
1 , & x\leq 0
\end{cases}
\] 显然在\(x=0\)处该函数不连续。
假设存在一个连续可微的解满足\(\dot{x}(t)=X(x(t))\)和\(x(0)=0\)。
然后\(\dot{x}(0)=X(x(0))=X(0)=1\),
于是,对于所有的充分小的时间\(t\),\(x(t)>0\)
并且\(\dot{x}(t)=X(x(t))=-1\), 这与\(t\mapsto \dot{x}(t)\)连续矛盾。
所以,不存在classical 解。
\end{quote}

下面的例子说明如果向量场不连续,那么(10)也可能存在经典解。

\begin{quote}
考虑向量场\(X:\mathbb{R}\to\mathbb{R}\) \[
X(x)=-\mathrm{sign}(x)=\begin{cases}
-1, & x>0,\\
0,   & x=0,\\
1,   & x<0,
\end{cases}
\] 唯一最大解为: \[
\begin{aligned}
&x(t)=x(0)-t, 
t\in [0,x(0)) 
&\textrm{if}
\ x(0)>0, \\
&x(t)=0,  t\in [0,\infty)
&\textrm{if}\ x(0)=0 
\\
&x(t)=x(0)+t,  t\in [0,-x(0))
&\textrm{if}\ 
x(0)<0,
\end{aligned}
\]
\end{quote}

下面例子说明连续但是不是单侧Lipschitz连续的向量场可能有多个classical解

\begin{quote}
考虑向量场\(X:\mathbb{R}\to\mathbb{R}\) \[
X(s)=\sqrt{|x|}
\] 这个向量场处处连续,并在\(\mathbb{R}/\{0\}\)局部Lipschitz连续,
但是在零处不局部连续,在零的领域也不单侧Lipschitz连续。
从\(x(0)=0\)开始,该动态系统有许多最大解,具体而言为:
对所有\(a>0\),\(x_a:[0,\infty)\to \mathbb{R}\),表达式为: \[
x_a(t)=\begin{cases}
0, & 0\leq t \leq a,\\
(t-a)^2/4, & t\geq a
\end{cases}
\]
\end{quote}

下面例子说明连续但是不是单侧Lipschitz连续的向量场只有一个classical解
\textgreater{} 考虑向量场\(X:\mathbb{R}\to\mathbb{R}\) \[
X(s)=\begin{cases}
-x \mathrm{log} x, & x>0\\
0,                            & x=0,\\
x \mathrm{log}(-x),& x<0,
\end{cases}
\] \textgreater{} 唯一最大解为: \[
\begin{aligned}
    &x(t)=-\exp(\log (-x(0)) \exp(t)), 
    &\textrm{if}\ x(0)<0 \\
    &x(t)=0, 
    &\textrm{if}\ x(0)=0 \\
    &x(t)=\exp(\log x(0) \exp(-t)), 
    &\textrm{if}\ x(0)>0
\end{aligned}
\]



\section{非光滑Lyapunov 方法(chark梯度)}


本节参照\parencite{shevitzLyapunovStabilityTheory1994,郑凯_基于Filippov微分包含解的非平滑控制系统},介绍非光滑微分方程Filippov解的定义及其稳定性的基本定理。

考虑向量微分方程
\begin{equation}\label{eq:nonsmooth diff}
    \dot{x}=f(x,t)
\end{equation}
其中函数 $f:\RR^p \times \RR \to \RR^p$ 是可测的(measurable),并且满足局部本性有界(essentially locally bounded)。
首先定义该方程的解。

\begin{definition}
    (Filippov解,\cite{filippovDifferentialEquationsDiscontinuous1988})
    如果一个关于时间$t$的函数$x(t)$在$[t_0,t_1]$上绝对连续,并且对于几乎(almost)所有$t\in[t_0,t_1]$满足
    \[
        \dot{x}\in \mathbb{K}[f](x,t),
        \quad 
        \mathbb{K}[f](x,t)=\bigcap_{\delta>0} \bigcap_{\mu(\bar{N})=0} \overline{co} f(B(x,\delta)\backslash \bar{N},t)
    \]
    那么称该函数$x(t)$为\eqref{eq:nonsmooth diff}的解。
    这里,
    $\bigcap_{\mu(\bar{N})=0}$表示所有Lebesgue测度为0的集合$\bar{N}$的交集,
    $\overline{co}(X)$表示$X$的凸闭包,$B(x,\delta)$表示以$x$为球心,$\delta$为半径的开球。
\end{definition}

\subsubsection{广义梯度与广义微分}

广义梯度和广义微分都是针对Lipschitz连续函数的,Lipschitz连续是连续的一种特殊情况。

\begin{definition}[广义方向导数]\cite{clarkeOptimizationNonsmoothAnalysis1990}
    若函数$V:\Omega\to\RR$在$x$附近是Lipschitz连续的,向量$v\in\Omega$,则函数$V$在$x$处沿$v$方向的广义方向导数为
    \begin{equation}
        V^o(x;v)=
        \limsup_{y\to x ,h\downarrow 0} 
        \frac{V(y+hv)-V(y)}{h}
    \end{equation}
    其中$y\in\Omega$.
\end{definition}
这里的广义方向导数在$x$处是关于$v$的正齐次可加函数,可作为支持函数确定一凸集\cite{clarkeOptimizationNonsmoothAnalysis1990},我们将这一凸集定义为函数$f(x)$在$x$处的广义梯度。

\begin{definition}[广义梯度]\cite{clarkeOptimizationNonsmoothAnalysis1990}
    若函数$V:\Omega\to\RR$在$x$附近是Lipschitz连续的,则函数$f$在$x$处的广义梯度定义为
    \begin{equation}
        \partial V(x)=
        \{\zeta \in\RR^n \big|  
        V^o(x;v)\geq \langle\zeta,v\rangle, \forall v\in\Omega\}
    \end{equation}
\end{definition}

\begin{theorem}[广义梯度的计算]
    \cite{clarkeOptimizationNonsmoothAnalysis1990}
    \label{thm:general gradient}
    对于一个局部Lipschitz连续的函数$V:\RR^p\times \RR\to \RR$,
    定义
    \begin{equation}
        \label{eq:general gradient}
        \partial V(x,t)=\overline{co}\{\lim_{y\to x}\nabla V(x)|
        y\notin S,
        y\notin \Omega_f\}
    \end{equation}
    为函数$V$在$(x,t)$处的广义梯度,
    其中,$S\subset \RR^n$, $\Omega_f\in\Omega$为$x$附近所有不可微点的集合。
\end{theorem}
定理\ref{thm:general gradient}可以看做是广义梯度的等价定义,除去所有梯度不存在的点,然后利用凸集来构造新的梯度集合。
若函数在$x$处本身就是可微的,则由梯度定义可知式\eqref{eq:general gradient}中的极限存在且唯一,这说明广义梯度也适用于平滑函数。
% 这一点与式(2-5)给出的KF[f](xt)的定义类似。

\begin{theorem}
    \cite{shevitzLyapunovStabilityTheory1994}
    令向量函数$x(t)$在区间$[t_0,t_1]$上是方程\eqref{eq:nonsmooth diff}的Filippov解,Lipschitz函数$V:\RR^n\times \RR \to \RR$是正则的,则函数$V(x(t),t)$几乎处处存在,并满足 
    \begin{equation}
        \frac{\dd }{\dd t} V(x(t),t)\overset{a.e.}{\in} \dot{\tilde{V}}(x,t)\triangleq
        \bigcap_{\xi \in \partial V}
        \xi^T \begin{bmatrix}
            \mathbb{K}[f](x,t)
            \\1
        \end{bmatrix}
    \end{equation}
\end{theorem}
此处,如果对于所有的$\psi$存在usual one-sided directional 导数$f'(x;\psi)$, 并且$f'(x;\psi)=f^o(x;\psi)$,其中
\begin{equation*}
    f^o(x;\psi)=\lim_{y\to x,t\downarrow 0 }\sup \frac{f(y+t\psi)-y}{t}
\end{equation*}
那么称$f(x,t):\RR^p \times \RR\to \RR$为正则函数(regular function)。

\begin{theorem}[非光滑系统LaSalle定理]\label{thm:LaSalle}
    \cite{shevitzLyapunovStabilityTheory1994} \textbf{Theorem} 3.2
    % Let $\Omega$ be a compact set such that every Filipov solution to the autonomous system $\dot{x} = f(x)$, 
    % $x(0) = x(t_0)$ starting in $\Omega$ is unique and remains in $\Omega$ for all $t \geq  t_0$. 
    % Let $V: \Omega \to R$ be a time independent regular function such that $v\leq 0$ for all $v\in\dot{\tilde{V}}$ 
    % (if $\dot{\tilde{V}}$ is the empty set then this is trivially satisfied). 
    % Define $S = \{x \in \Omega | 0 \in \dot{\tilde{V}}\}$. 
    % Then every trajectory in $\Omega$ converges to the largest invariant set, $M$, in the closure of $S$.
    设 $\Omega$ 是一个紧集,使得对于所有从 $\Omega$ 出发的自治系统 $\dot{x} = f(x)$,$x(0) = x(t_0)$ 的每一个 Filippov 解都是唯一的,并且在所有 $t \geq t_0$ 的时间内都保持在 $\Omega$ 中。
    设 $V: \Omega \to R$ 是一个与时间无关的正则函数,使得对于所有 $v \in \dot{\tilde{V}}$,都有 $v \leq 0$(如果 $\dot{\tilde{V}}$ 是空集,则这一条件自然满足)。
    定义 $S = \{x \in \Omega | 0 \in \dot{\tilde{V}}\}$。
    那么,$\Omega$ 中的每一条轨迹都会收敛到 $S$ 的闭包中的最大不变集 $M$。
\end{theorem}

