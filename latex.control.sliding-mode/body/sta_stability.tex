除了本章所引用参考文献外,网络上也有一些关于STA算法的讨论,主要摘录如下:
\begin{enumerate}
	\item 知乎: 关于超螺旋滑模算法的完整证明过程 \url{https://zhuanlan.zhihu.com/p/672355794}
	\item CSDN: 超螺旋滑模控制详细介绍(全网独家) \url{https://blog.csdn.net/qq_41811966/article/details/134869940}
\end{enumerate}

\subsection{稳定性条件概览}

为了尽可能地兼顾一般性和行文简便,这里将\eqref{eq sta s}写为:
\begin{equation}
	\begin{cases}
		\dot{s}=-k_1 \sig{s}^{1/2}+w+\varrho_1\\
		\dot{w}=-k_2 \sgn{s}+\varrho_2
	\end{cases}
\end{equation}
这里$\varrho_1,\varrho_2$分别为第一通道和第二通道的扰动。
事实上,这里的第一通道扰动可以合并到第二通道,
即记$w'=w+\varrho_1$这样$\dot{w}'=\dot{w}+\dot{\varrho_1}=-k_2 \sgn{s}+\varrho_2+\dot{\varrho_1}$.
于是,系统可以转化为$\varrho_1'=0$, $\varrho_2'=\varrho_2+\dot{\varrho}_1$.
因此,表\ref{table sta gain selection}给出了已有文献对增益选择的要求。

\begin{table}[!htb]
	\centering
	\begin{tabular}{c|c|c}
		\hline
		code & 参数条件  & 出处 \\\hline
		Levant1998 &
		$k_1\geq 2\sqrt{L\frac{k_2+L}{k_2-L}}$, $k_2>L$ &
		\makecell{
			条件由\cite[eq 6]{levantRobustExactDifferentiation1998}给出并使用几何方法证明,\\
			但是文中也说到,这个条件只是充分不是必要的,\\
			比如$k_1=1\sqrt{L}$,
			$k_2=1.1L$ 或者 \\
			$k_1=0.5*\sqrt{L}$,
			$k_2=4L$
			也有较好的收敛效果。
		}
		\\\hline 
		Moreno2012 &
			$k_1>2\sqrt{k_2-\sqrt{k_2^2-L^2}}$, $k_2>L$
		&
			\makecell{
			原文为算法\ref{algo sta moreno}\cite[algorithm 1]{morenoStrictLyapunovFunctions2012},\\
			由\cite{seeberStabilityProofWellestablished2017}推导得到本条件
		}
		\\\hline
		Moreno2014 &
		$k_1>\sqrt{\frac{2}{k_1-L}}$,$k_2>L$
		&
		\cite{morenoStrictLyapunovFunctions2014}
		\\\hline
		Seeber2017&
		$k_1>\sqrt{k_2+L}$,$k_2>L$ &
		\makecell{
		\cite{seeberStabilityProofWellestablished2017}\\
		\cite{colottiNewConvergenceConditions2022}拓展到带有时变增益的情形
		}
		\\\hline
		Seeber2018 &
		充要条件,较复杂见参考文献
		&
		几何证明,\cite{seeberNecessarySufficientStability2018}
		\\\hline
		%----------------------------------------
		Feng2020 &
		$k_1 >0$,
		$k_2>L$& 
		引自多体文献\cite{fengFinitetimeDistributedConvex2020},
		%文献中给出的$k_1$范围应为笔误。
		\\\hline
		Chen2024 &
		$k_1 >\sqrt{\frac{-2L^2+5k_2L+11k_2^2}{k_2-L}} $,
		$k_2>L$&
		改写自多体文献\cite{chenDistributedFinitetimeDifferentiator2024}
		\\\hline
	\end{tabular}
	\caption{STA收敛条件对比,在第一通道无扰动$\varrho_1=0$,$|\varrho_2|\leq L$的参数对比}
	\label{table sta gain selection}
\end{table}
从表中可以发现,$k_1$的条件与$k_2$和$L$都有关系,直观可以从图\ref{fig sta gain selection}看出取值范围的大致界限。
图中,可以明显看到当$k_2$较大的时候,$k_1$可以取到小于$k_2$的值。

\begin{figure}[!htb]
	\centering
	\includegraphics[height=6cm]{simulation/out/k1-k2}
	\caption{不同文献中关于STA算法增益选取的讨论,图中x标记为\ref{fig basic sta}所使用的稳定的参数$L=0.1,k_1=0.18,k_2=0.2$,可以发现目前文献中的参数选取标准仍然是保守的。}
	\label{fig sta gain selection}
\end{figure}

\begin{algo}\label{algo sta moreno}
	当 $\varrho_2=0$ 时,$k_1>0$并且 $k_2>0$ ,
	当 $\varrho_2>0$ 时,通过以下规则确定
	\begin{enumerate}
		\item  选择正常数$(\beta,\gamma)$ 使得 $0<\beta<1,\gamma>1$
		\item 找到满足下列不等式的正常数 $\chi,\alpha$
		\begin{equation}
			\chi-\frac{2}{\gamma}\alpha
			>\alpha^2-\beta (1+\chi)\alpha+\frac14 (1+\chi)^2
		\end{equation}
		\item 选取增益参数为
		\begin{equation}
			k_1=\chi \sqrt{\frac{2\gamma}{(1-\beta)\alpha}}\sqrt{L},\quad 
			k_2=\frac{\beta+1}{1-\beta}L
		\end{equation}
	\end{enumerate}
\end{algo}

\section{严格Lyapunov函数的稳定性证明}

%本节展示多维情况下的证明方法,
%内容主要来自\cite{nageshMultivariableSupertwistingSliding2014,basinMultivariableContinuousFixedtime2017}.文中Remark 3 说到这里的$k_2$ 主要为了抑制扰动,只要$k_2>\Delta$即可抑制住扰动,所以该证明还是太保守。


\begin{proposition}
	\cite[Proposition 1]{morenoStrictLyapunovFunctions2014}
	考虑多输入多输出的STA系统,其中$s,w\in\RR^n$
	\begin{equation}
		\dot{s}=-k_1 \sig{s}^{1/2}+w,\quad 
		\dot{w}=-k_2 \sgn{s}+\delta(t),\quad 
		|\delta(t)|<\Delta.
	\end{equation}
	其中, 增益的选择为$k_2>\Delta$,$k_1>\sqrt{2k_1}$.
\end{proposition}

\begin{proof}
	
	选取Lyapunov 函数为
	\begin{equation}
		\label{eq V zeta}
		V=\zeta^T P \zeta + (\gamma_1 +\gamma_2)k_2 |s|\quad 
		P=\frac12 \begin{bmatrix}
			\gamma_2 k_1^2  & -\gamma_2 k_1\\
			-\gamma_2 k_1 & (\gamma_1+\gamma_2)\\
		\end{bmatrix}
		\quad 
		\zeta =\begin{bmatrix}
			\sig{s}^{1/2}\\
			w
		\end{bmatrix}
	\end{equation}
	这里把$w$作为一个状态,这样做不太合适,因为我们可以从仿真中看到,$w$其实并不是趋于零得量,
	他是某种意义上得对$\delta(t)$的一个有偏差的估计量,所以我们需要使用另一种表达形式:
	\begin{equation}
		V=\chi^T R \chi + (\gamma_1 +\gamma_2)k_2 |s|\quad 
		R=\frac12 \begin{bmatrix}
			\gamma_1 k_1^2  & -\gamma_1 k_1\\
			-\gamma_1 k_1 & (\gamma_1+\gamma_2)\\
		\end{bmatrix}\quad 
		\chi=\begin{bmatrix}
			\sig{s}^{1/2}\\
			\sig{s}^{1/2}-w
		\end{bmatrix}
	\end{equation}
	在求导的时候,实际上还是使用\eqref{eq V zeta}的展开形式更加直观,所以这里将它展开为:
	\begin{gather*}
		V=V_1+V_2\\
		V_1=\frac12 \gamma_2 k_1^2 |s|
		+ \frac12(\gamma_1+\gamma_2) w^2
		-\gamma_2 k_1 \sig{s}^{1/2}w
		\\
		V_2=(\gamma_1+\gamma_2)k_2 |s|
	\end{gather*}
	求导得:
	\begin{align*}
		\dot{V}_1
		&=\tR{\frac12 \gamma_2 k_1^2 \sgn{s}\dot{s}}
		+(\gamma_1+\gamma_2) w \dot{w}
		-\gamma_2 k_1 \sig{s}^{1/2}\dot{w}
		\tR{-\gamma_2 k_1 \frac12 |s|^{-1/2}w\dot{s}}
		\\
		&=\tR{\frac12 \gamma_2 k_1 |s|^{-1/2} (k_1\sig{s}^{1/2}-w)\dot{s}}
		+\left((\gamma_1+\gamma_2) w -\gamma_2 k_1 \sig{s}^{1/2}\right)\dot{w}
		\\
		&=\tR{-\frac12 \gamma_2 k_1 |s|^{-1/2} |\chi_2|^2}
		+\left(-(\gamma_1+\gamma_2) \chi_2 +\gamma_1 k_1 \sig{s}^{1/2}\right)(-k_2\sgn{s}+\delta(t))
	\end{align*}
	\begin{align*}
		\dot{V}_2=(\gamma_1+\gamma_2)k_2  \sgn{s}\dot{s}
		=-(\gamma_1+\gamma_2)k_2  \sgn{s}\chi_2
	\end{align*}
	这里注意要把关于$w$的部分都作为$\chi_2$的一部分提出来
	\begin{align*}
		\dot{V}
		&=\tR{-\frac12 \gamma_2 k_1 |s|^{-1/2} |\chi_2|^2}
		-\gamma_1 k_1 (k_2-\sgn{s} \delta(t))|s|^{1/2}
		\tB{-(\gamma_1+\gamma_2)\chi_2 \delta(t)}
		\\
		&=\tR{-\frac12 \gamma_2 k_1 |s|^{-1/2} |\chi_2|^2}
		-\gamma_1 k_1 (k_2-\Delta)|s|^{1/2}
		\tB{+(\gamma_1+\gamma_2)|\chi_2| \Delta}
	\end{align*}
	对\tR{红色}部分使用杨氏不等式可得,对于任意的$\mu>0$有
	\begin{equation}
		|s|^{-1/2}\chi_2^2\geq 
		-\mu^{-1}|s|^{1/2}
		+2{\mu^{-1/2}}|\chi_2|
	\end{equation}
	于是,可以得到表达式
	\begin{equation}
		\dot{V}\leq 
		-\left(
			\gamma_1 (k_2-\Delta)-\frac12 \gamma_2 \mu^{-1}
		\right)
		k_1 |s|^{1/2}
		-\left( \gamma_2 \mu^{-1/2}k_1-(\gamma_1+\gamma_2)\Delta\right)
		|\chi_2|
	\end{equation}
	为了保证V导数负定,首先我们的$\gamma_1,\gamma_2$是可以任意选取的,所以这里选择
	\begin{equation*}
		\gamma_2=\frac{2\mu}{\nu }(k_2 -\Delta) \gamma_1,\nu>1
	\end{equation*}
	于是,我们有
	\begin{equation}
		\dot{V}\leq 
		-\frac{\nu-1}{\nu}\left(\gamma_1 (k_2-\Delta)\right)
		k_1 |s|^{1/2}
		-\left( \gamma_2 \mu^{-1/2}k_1-(\gamma_1+\gamma_2)\Delta\right)
		|\chi_2|
	\end{equation}
	基于此,我们可以指导要使得整个系统的V导数负定,只需要
	\begin{equation}
		k_2>\Delta,\quad 
		k_1>\frac{(\gamma_1+\gamma_2)\Delta\mu^{1/2}}{\gamma_2}
		=\frac{1+\frac{2\mu}{\nu }(k_2 -\Delta) }{\frac{2\mu}{\nu }(k_2 -\Delta) }\Delta \mu^{1/2}
	\end{equation}
	因为$\mu$可以任意选取,因此,$k1$只需要大于右侧最小值$\min_{\mu>0} \frac{1+\frac{2\mu}{\nu }(k_2 -\Delta) }{\frac{2\mu}{\nu }(k_2 -\Delta) }\Delta \mu^{1/2}$,
	其最小值点在$\mu=\mu^*=\frac{\nu}{2(k_2-\Delta)}$,
	于是有
	\begin{equation}
		k_1> \sqrt{\frac{2\nu}{k_2 -\Delta}}\Delta
	\end{equation}
	
	
	\begin{quote}
		微分同胚英文是“diffeomorphism”。理解微分同胚可以从以下方面入手:
		
		从几何角度看,两个流形(比如曲线、曲面等几何对象)如果是微分同胚的,那么它们在“光滑”的意义下是一样的。它是一种特殊的同胚映射。同胚简单说就是一种连续的双射(一一对应),并且它的逆映射也是连续的。而微分同胚在此基础上还要求这个映射和它的逆映射都是可微的(光滑的)。
		
		例如,一个圆和一个椭圆在拓扑意义下是同胚的,它们可以通过连续变形从一个得到另一个。如果这个变形过程是光滑的,而且逆变形过程也是光滑的,那么它们就是微分同胚的。
	\end{quote}
	
	
\end{proof}





\section{高阶滑模和高阶微分器}

\cite{morenoSurveyHighorderSlidingmode2023}给出了高阶情况的Lyapunov法证明方法。
同时该文献将算法拓展到了更加一般的情况(幂指数从1/2变为0到1的数,同时添加了线性环节)。

