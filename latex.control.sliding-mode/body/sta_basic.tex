考虑如下简单的二阶双积分器系统:
\begin{equation}
	\begin{cases}
		\dot{x}_{1} = x_{2} \\
		\dot{x}_{2} = u + d(t)
	\end{cases}
\end{equation}
其中,$x=[x_1,x_2]^T \in \mathbb{R}^{2}$表示系统的状态向量,
$u\in \mathbb{R}$是控制输入向量,
$d(x,t)\in \mathbb{R}$代表外部干扰以及未建模动态等不确定性因素。

定义滑模面函数为:
\begin{equation}
	s=c x_1+x_2
\end{equation}
其中,$c \in \mathbb{R}$为合适的常数,使得滑模面满足一定的可达性与稳定性要求。

对滑模变量求导得,$\dot{s}=c x_2+ u+d(t)$
超螺旋滑模控制律通常设计为:
\begin{equation}\label{eq sta s}
	\begin{cases}
		\dot{s}=-k_1  \sig{s}^{1/2}+w\\
		\dot{w}=-k_2 \sgn{s}-\dot{d}(t)
	\end{cases}
\end{equation}
式中,$k_1 > 0$和$k_2 > 0$为待确定的控制增益参数,$\sgn{s}$是符号函数。
一定程度上,该算法可以视为一种PI控制。
该控制律通过幂为$1/2$的比例环节和幂为$0$的积分环节综合作用,
驱使系统状态在有限时间内到达滑模面,并沿着滑模面渐近稳定,同时抑制外部干扰的影响。
(注:$\sig{s}^0=\sgn{s}$)


最后的滑模控制率为:
\begin{equation}
	u = -k_1 \sig{s}^{1/2} + w + d(t) -c x_2
\end{equation}

这里以一个二阶单输入单输出(SISO)系统为例展示超螺旋滑模控制的仿真实现,
其中,$d(t) = 0.1\sin(2t)$表示外部干扰。
取滑模参数$c=1$,格库塔四阶方法仿真代码见\href{https://github.com/xsro/xsro.github.io/blob/zola/typst/sliding-mode-control/simulation/sta.py}{sta.py},
如果需要更加精确的仿真可以参见\cite{livneProperDiscretizationHomogeneous2014}.
仿真效果如图\ref{fig basic sta}所示。
可以发现系统稳定后$w(t)+d(t)=0$,系统的积分部分达到了观测扰动的作用。
\begin{figure}
	\centering
	\includegraphics[height=6cm]{simulation/out/sta_0.18_0.2}
	\includegraphics[height=6cm]{simulation/out/sta_0.17_0.2}
	\caption{STA算法rk4仿真,$k_1=0.18$(左) $k_1=0.17$(右),
		 $k_2=0.2$, $d=0.1 \sin(t)$.}
	\label{fig basic sta}
\end{figure}
