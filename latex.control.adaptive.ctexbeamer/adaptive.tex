\documentclass{beamer}
\usetheme{Madrid} % 简洁清晰的Beamer主题
\usecolortheme{whale} % 沉稳配色,适配学术场景
\usepackage{ctex} % 中文支持
\usepackage{amsmath,amssymb,graphicx} % 公式与图形支持
\usepackage{hyperref} % 超链接

% 字体配置
\setCJKmainfont{SimSun} % 中文正文字体(宋体)
\setCJKsansfont{Microsoft YaHei} % 中文标题字体(微软雅黑)

% 课件基本信息
\title{参数估计与充分丰富信号}
\subtitle{自适应控制的核心方法与收敛条件}
\author{基于Alessandro Astolfi课程文档整理}
\institute{自动化学院 控制理论与应用教研室}
\date{\today}

\begin{document}

% ---------------------- 第1页:封面 ----------------------
\begin{frame}
  \titlepage
  \note{1分钟:封面页快速带过,明确课件主题是“参数估计”与“充分丰富信号”}
\end{frame}

% ---------------------- 第2页:目录 ----------------------
\begin{frame}{目录}
  \tableofcontents[sectionstyle=show/shaded, subsectionstyle=show/show/shaded]
  \note{1分钟:梳理课件逻辑,让听众明确核心模块}
\end{frame}

% ---------------------- 第3页:参数估计的基本概念 ----------------------
\section{参数估计:自适应控制的“感知核心”}
\begin{frame}{为什么需要参数估计?}
  \begin{block}{核心问题:系统参数未知性}
    自适应控制的前提是“处理不确定性”,而**参数未知**是最常见的不确定性形式,例如:
    - 线性系统:$\dot{x}=Ax+Bu$中矩阵$A,B$的元素未知;
    - 非线性系统:$\dot{x}=a_0f(x,t)+b_0g(x,t)+u$中系数$a_0,b_0$未知。
  \end{block}

  \vspace{0.5cm}
  \begin{block}{参数估计的本质}
    \begin{center}
      \textbf{观测响应 $y(t)$ vs 模型响应 $\hat{y}(\hat{\theta},t)$} $\to$ \textbf{更新参数估计 $\hat{\theta}$}
    \end{center}
    目标:使估计误差$\epsilon=y-\hat{y}$渐近收敛,最终$\hat{\theta}\to\theta$(真实参数)。
  \end{block}
  \note{3分钟:结合文档1-821~838页,讲清参数估计的背景与核心逻辑}
\end{frame}

% ---------------------- 第4页:参数估计的核心框架 ----------------------
\begin{frame}{参数估计的三步骤框架}
  \begin{enumerate}
    \item[Step 1:\textbf{模型参数化}]
      将系统表示为“可测信号+未知参数”的形式,文档中核心模型包括:
      - 线性参数模型:$z=\theta^\top\phi$($\phi$为可测回归向量,如滤波后的$u,y$);
      - 双线性参数模型:$y=k_0(\theta^\top\phi+z_0)$($k_0$为未知高频增益,$\theta$为未知参数向量)。

    \item[Step 2:\textbf{设计更新律}]
      基于误差信号$\epsilon$设计$\dot{\hat{\theta}}$,保证$\hat{\theta}$有界且$\epsilon$收敛。

    \item[Step 3:\textbf{输入信号设计}]
      选择“充分丰富的输入$u$”,确保参数估计能收敛到真实值(核心:解决“信息不足”问题)。
  \end{enumerate}

  \vspace{0.3cm}
  \begin{figure}
    % \includegraphics[width=0.7\textwidth]{param_est_flow.pdf} % 替换为参数估计流程图(如文档1-838页逻辑)
    \caption{参数估计的闭环逻辑}
  \end{figure}
  \note{4分钟:基于文档1-696~813页参数化内容,拆解框架,引出“输入设计”的必要性}
\end{frame}

% ---------------------- 第5-6页:常用参数估计方法 ----------------------
\section{常用参数估计方法}
\begin{frame}{1. 梯度法(Gradient Identifiers)}
  \begin{block}{核心模型:线性参数化$z=\theta^\top\phi$}
    估计误差定义:$\epsilon=z-\hat{z}=z-\hat{\theta}^\top\phi=-\tilde{\theta}^\top\phi$($\tilde{\theta}=\hat{\theta}-\theta$)。
  \end{block}

  \begin{block}{梯度更新律设计}
    最小化瞬时代价$J(\hat{\theta})=\frac{1}{2}\epsilon^2$,由梯度下降得:
    \[
    \dot{\hat{\theta}}=\Gamma\phi\epsilon
    \]
    其中$\Gamma=\Gamma^\top>0$为自适应增益(调节参数更新速度)。
  \end{block}

  \begin{block}{稳定性分析(Lyapunov方法)}
    取Lyapunov函数$V=\frac{1}{2}\tilde{\theta}^\top\Gamma^{-1}\tilde{\theta}$,则:
    \[
    \dot{V}=\tilde{\theta}^\top\Gamma^{-1}\dot{\tilde{\theta}}=\tilde{\theta}^\top\phi\epsilon=-\epsilon^2\leq0
    \]
    结论:$\epsilon\in L_2\cap L_\infty$,$\hat{\theta}\in L_\infty$(参数有界)。
  \end{block}
  \note{4分钟:详细讲解文档1-1048~1067页梯度法,突出稳定性分析核心}
\end{frame}

\begin{frame}{2. SPR设计与DREM方法}
  \begin{block}{SPR设计(Strictly Positive Real)}
    核心思想:对参数模型滤波,使传递函数$W(s)L(s)$为\textbf{严格正实},结合无源性定理:
    - 误差系统:$\epsilon=W(s)L(s)(-\tilde{\theta}^\top\phi)$(严格无源);
    - 参数更新:$\dot{\hat{\theta}}=\Gamma\phi\epsilon$(无源子系统);
    - 互联后:$\epsilon\to0$,若$\phi$持续激励则$\hat{\theta}\to\theta$(文档1-1013~1037页)。
  \end{block}

  \vspace{0.3cm}
  \begin{block}{DREM方法(动态回归扩展混合)}
    解决“参数耦合”问题:通过滤波与矩阵伴随运算,将多参数估计解耦为 scalar 估计:
    \[
    \tilde{\mathcal{Z}}_i=\det\Phi\cdot\theta_i \implies \dot{\hat{\theta}}_i=\gamma_i\det\Phi(\tilde{\mathcal{Z}}_i-\det\Phi\hat{\theta}_i)
    \]
    优势:每个参数独立更新,保证$|\tilde{\theta}_i(t)|$非增(文档1-1093~1111页)。
  \end{block}
  \note{4分钟:简化文档中SPR和DREM的复杂推导,聚焦核心优势与应用场景}
\end{frame}

% ---------------------- 第7-9页:充分丰富信号的定义与意义 ----------------------
\section{充分丰富信号:参数收敛的关键}
\begin{frame}{为什么需要“充分丰富信号”?}
  \begin{block}{反例:信息不足的输入无法估计参数}
    以 scalar 系统$\dot{x}=-ax+bu$($a,b$未知)为例:
    - 若$u=0$:$x(t)=e^{-at}x(0)$,仅含$a$的信息(且指数衰减),无法估计$b$;
    - 若$u=u_0$(常数):$x(t)=e^{-at}(x(0)-\frac{b}{a}u_0)+\frac{b}{a}u_0$,仅能估计$\frac{b}{a}$(比例),无法分离$a,b$;
    - 若$u=\sin\omega_0t$(正弦信号):$x(t)$含稳态正弦分量,可从幅值/相位提取$a,b$信息(文档1-955~966页)。
  \end{block}

  \vspace{0.3cm}
%   \begin{conclusion}
    输入信号$u$需携带“足够多的信息”,才能使参数估计$\hat{\theta}$收敛到真实值$\theta$——这就是“充分丰富信号”的核心作用。
%   \end{conclusion}
  \note{4分钟:用文档1-955~966页的 scalar 例子,直观说明信号丰富性的必要性}
\end{frame}

\begin{frame}{充分丰富信号的定义}
  \begin{block}{定义1:基于频率数量(文档1-970~974页)}
    信号$u:\mathbb{R}^{\geq0}\to\mathbb{R}$是\textbf{n阶充分丰富}的,若其包含至少$\frac{n}{2}$个不同频率。\\
    例:$u(t)=\sum_{i=1}^m A_i\sin\omega_i t$($A_i\neq0$,$\omega_i\neq\omega_k$),则$u$是$2m$阶充分丰富的。
  \end{block}

  \vspace{0.3cm}
  \begin{block}{定义2:基于持续激励(Persistent Excitation,文档1-988~990页)}
    信号$u:\mathbb{R}^{\geq0}\to\mathbb{R}^m$是\textbf{持续激励}的,若存在$t_0,T>0$和$\alpha>0$,使得:
    \[
    \frac{1}{T}\int_t^{t+T} u(\tau)u^\top(\tau)d\tau \geq \alpha I>0 \quad \forall t\geq t_0
    \]
    物理意义:信号在任意时间窗内的“能量分布”不趋于零,持续提供信息。
  \end{block}
  \note{4分钟:准确呈现文档1-970~990页的两个核心定义,用例子辅助理解}
\end{frame}

\begin{frame}{充分丰富信号与参数收敛的定理}
  \begin{theorem}[核心定理,文档1-991~1006页]
    考虑线性系统$\dot{x}=A_p x+B_p u$($A_p$渐近稳定,$(A_p,B_p)$能控),参数估计采用并行模型(P)或串并行模型(SP):
    1. 若$u$是$(n+1)$阶充分丰富信号($n$为系统阶数),则$\hat{A}_p\to A_p$,$\hat{B}_p\to B_p$(指数收敛);
    2. 若多输入$u_i$均为$(n+1)$阶充分丰富且频率互不重叠,则参数估计同样指数收敛。
  \end{theorem}

  \vspace{0.3cm}
  \begin{corollary}[回归向量与持续激励]
    设$\phi=H(s)u$($H(s)$稳定真有理),若$u$是$n$阶充分丰富信号,且$H(j\omega_1),\dots,H(j\omega_n)$线性无关,则$\phi$是持续激励的——最终保证$\hat{\theta}\to\theta$。
  \end{corollary}

  \vspace{0.2cm}
  \begin{figure}
    %\includegraphics[width=0.6\textwidth]{signal_richness.pdf} % 替换为“充分/不充分信号的频率对比图”
    \caption{充分丰富信号(多频率)与非充分信号(单频率)的频谱差异}
  \end{figure}
  \note{5分钟:讲解文档1-991~1006页的核心定理,建立“信号丰富性→持续激励→参数收敛”的逻辑链}
\end{frame}

% ---------------------- 第10-11页:实例验证 ----------------------
\section{实例: scalar 系统的参数估计与信号影响}
\begin{frame}{实例设置(文档1-1113~1117页)}
  被控系统:$\dot{x}=-ax+bu$,真实参数$a=0.4$,$b=0.4$;\\
  参数估计方法:梯度法 vs DREM方法;\\
  输入信号对比:
  \begin{enumerate}
    \item 信号1:$u(t)=10$(常数,1阶,非充分丰富);
    \item 信号2:$u(t)=10\sin\left(\frac{5t}{2}\right)$(单频率,2阶,充分丰富)。
  \end{enumerate}

  \begin{block}{DREM滤波设置}
    选择两个不同时间常数的滤波器:
    \[
    H_1(s)=\frac{1}{s+1}, \quad H_2(s)=\frac{2}{s+2}
    \]
    生成解耦的回归信号$\mathcal{Z}$与$\Phi$,实现$a,b$的独立估计。
  \end{block}
  \note{3分钟:基于文档1-1113~1117页的Assignment 2,明确实例参数与对比方案}
\end{frame}

\begin{frame}{实例结果分析}
  \begin{block}{1. 输入信号对收敛性的影响}
    - 信号1(常数$u=10$):$\hat{a}$收敛到近似值,但$\hat{b}$无法收敛(仅能估计$\frac{b}{a}$);
    - 信号2(正弦$u=10\sin(2.5t)$):$\hat{a}\to0.4$,$\hat{b}\to0.4$(指数收敛,因$u$是2阶充分丰富)。
  \end{block}

  \begin{block}{2. 估计方法性能对比}
    \begin{tabular}{|c|c|c|}
      \hline
      性能指标 & 梯度法 & DREM方法 \\
      \hline
      参数耦合性 & 耦合($\hat{a},\hat{b}$相互影响) & 解耦(独立更新) \\
      \hline
      瞬态响应 & 超调较大 & 无超调($|\tilde{\theta}_i|$非增) \\
      \hline
      抗噪声性 & 一般(易受$u,y$噪声干扰) & 更强(滤波抑制噪声) \\
      \hline
    \end{tabular}
  \end{block}

  \vspace{0.2cm}
  \begin{figure}
    %\includegraphics[width=0.5\textwidth]{est_result.pdf} % 替换为参数估计收敛曲线(如$\hat{a}(t),\hat{b}(t)$)
    \caption{正弦输入下的参数估计收敛曲线}
  \end{figure}
  \note{3分钟:结合文档实例结论,用表格和曲线直观对比,强化“充分丰富信号+合适方法”的重要性}
\end{frame}

% ---------------------- 第12-13页:总结与问答 ----------------------
\section{总结与工程启示}
\begin{frame}{核心内容总结}
  \begin{enumerate}
    \item[1. 参数估计的核心]
      基于“误差驱动更新”,常用方法各有侧重:
      - 梯度法:简单易实现,适合线性参数模型;
      - SPR设计:适用于仅测输入输出的场景;
      - DREM:解耦多参数估计,鲁棒性强。

    \item[2. 充分丰富信号的关键]
      - 定义:$n$阶信号需$\geq n/2$个不同频率,或满足持续激励条件;
      - 作用:是参数估计收敛到真实值的\textbf{必要条件}(无则仅能保证参数有界)。

    \item[3. 工程设计原则]
      - 输入设计:优先选择多频率组合信号(如$\sum A_i\sin\omega_i t$);
      - 方法选择:噪声小时用梯度法,多参数时用DREM,仅I/O测量时用SPR设计。
  \end{enumerate}
  \note{3分钟:梳理核心知识点,建立“方法-信号-应用”的关联}
\end{frame}

\begin{frame}{问答与讨论}
  \begin{block}{待思考的工程问题}
    1. 若系统参数时变(如$a(t)=1+0.1\sin t$),充分丰富信号是否仍能保证参数估计跟踪?\\
    2. 如何在“控制性能”与“信号丰富性”间平衡(如避免输入过大影响系统运行)?\\
    3. 对于非线性系统,充分丰富信号的定义是否需要调整?
  \end{block}

  \vspace{1cm}
  \begin{center}
    \Huge{谢谢聆听!}\\
    \vspace{0.5cm}
    \Large{欢迎提问与交流}
  \end{center}
  \note{4分钟:提出文档相关的延伸问题,引导讨论,结束课件}
\end{frame}

\end{document}