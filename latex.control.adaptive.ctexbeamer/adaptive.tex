\documentclass{beamer}
% 引用设置
\usepackage[backend=biber,style=authoryear]{biblatex}
\addbibresource{body/refs.bib} 

\usetheme{Madrid} % 简洁清晰的Beamer主题
\usecolortheme{whale} % 沉稳配色,适配学术场景
\usecolortheme{seahorse}
\usefonttheme{professionalfonts}

% 加载必要宏包
\usepackage{amsmath,amssymb,graphicx}
% 定义常用符号
\newcommand{\R}{\mathbb{R}} % 实数集
\newcommand{\abs}[1]{\left| #1 \right|} % 绝对值
\newcommand{\intlim}[2]{\int_{#1}^{\infty} #2 \, dt} % 无穷积分
\newcommand{\intt}[2]{\int_{0}^{#1} #2 \, d\tau} % 从0到t的积分
% 定义常用定理
\newtheorem{remark}{\textbf{Remark}} 

\usepackage{ctex} % 中文支持
\usepackage{hyperref} % 超链接
\usepackage{subcaption}
\newcommand{\tB}{\textcolor{blue}}


% 字体配置
\setCJKmainfont{SimSun} % 中文正文字体(宋体)
\setCJKsansfont{Microsoft YaHei} % 中文标题字体(微软雅黑)

\AtBeginSection[]{
  \begin{frame}
  \vfill
  \centering
  \begin{beamercolorbox}[sep=8pt,center,shadow=true,rounded=true]{title}
    \usebeamerfont{title}\insertsectionhead\par%
  \end{beamercolorbox}
  \vfill
  \end{frame}
}


% 课件基本信息
\title{自适应控制专题学习}
\subtitle{现代控制理论}
\author{刘成}
\institute{中山大学 航空航天学院}
\date{\today}

\begin{document}





% ---------------------- 第1页:封面 ----------------------
\begin{frame}
  \titlepage
  \note{1分钟:封面页快速带过,明确课件主题是“参数估计”与“充分丰富信号”}
\end{frame}

\begin{frame}
  \frametitle{本ppt主要涉及}
  \begin{enumerate}
    \item Barbalat's Lemma \footnote{\url{http://ele.aut.ac.ir/~abdollahi/Lec_4_N11.pdf}}
    \item 未知参数自适应控制 \footnote{自适应控制与持续激励条件 \url{https://zhuanlan.zhihu.com/p/9450457498}}  $\dot{x}=w(t)\mu +u$ 其中 $\mu$ 未知固定
    \item 未知控制方向自适应控制 \footnote{未知增益自适应与Nussbaum函数 \url{https://zhuanlan.zhihu.com/p/9910889013}}  $\dot{x}=gu$ 其中$g$ 未知固定
    \item 系统参数估计\footnote{自适应控制中的持续激励条件怎么理解? \url{https://www.zhihu.com/question/266601777/answer/375458125}} $y(t)=\phi(t)^T \theta$
    \item PE条件 (persistently exciting/sufficiently enough)
  \end{enumerate}
\end{frame}

% ---------------------- 第2页:目录 ----------------------
\begin{frame}{目录}
  \tableofcontents[]
  \note{1分钟:梳理课件逻辑,让听众明确核心模块}
\end{frame}


\section{引言}
\subsection{什么是自适应控制}

\begin{frame}{韦氏词典对“适应”与“适应性”的定义}
	\begin{itemize}
		\item According to the Webster’s dictionary, to \textbf{adapt} means:
		\begin{itemize}
			\item to adjust oneself to particular conditions
			\item to bring oneself in harmony with a particular environment
			\item to bring one’s acts, behaviour in harmony with a particular environment
		\end{itemize}
		\item 【中文翻译】根据韦氏词典,“adapt”(适应)的含义是:
		\begin{itemize}
			\item 使自己适应特定条件
			\item 使自己与特定环境相协调
			\item 使自己的行为举止与特定环境相协调
		\end{itemize}
	\end{itemize}
	\begin{itemize}
		\item According to the Webster’s dictionary, \textbf{adaptation} means:
		\begin{itemize}
			\item adjustment to environmental conditions
			\item alteration or change in form or structure to better fit the environment
		\end{itemize}
		\item 【中文翻译】根据韦氏词典,“adaptation”(适应性)的含义是:
		\begin{itemize}
			\item 对环境条件的调整
			\item 为更好地适应环境而在形式或结构上的改变或调整
		\end{itemize}
	\end{itemize}
\end{frame}



\begin{frame}{自适应控制器的定义与理论核心}
	\begin{itemize}
		\item An adaptive controller is a \textbf{fixed-structure} controller with \textbf{adjustable parameters} and a mechanism for \textbf{automatically} adjusting those parameters
		\item 【翻译】自适应控制器是一种\textbf{固定结构}的控制器,具有\textbf{可调参数}以及自动调整这些参数的机制
		\item In this sense, an adaptive controller is one way of dealing with \textbf{parametric uncertainty}
		\item 【翻译】从这个意义上说,自适应控制器是应对\textbf{参数不确定性}的一种方法
		\item Adaptive control theory essentially deals with finding parameter adjustment algorithms that guarantee global \textbf{stability and convergence}
		\item 【翻译】自适应控制理论本质上是研究寻找能保证\textbf{全局稳定性和收敛性}的参数调整算法
	\end{itemize}
\end{frame}


\begin{frame}
	\frametitle{什么是自适应控制}
	Adaptive Control covers a set of techniques which provide a systematic approach for
	automatic adjustment of controllers in real time, in order to achieve or to maintain
	a desired level of control system performance when \textbf{the parameters of the plant
	dynamic model are unknown and/or change in time.}
	
	[1] Landau I D, Lozano R, M’Saad M, et al. Adaptive Control: Algorithms, Analysis and Applications[M]. London: Springer London, 2011.
	
	\vspace{1cm}
	
	\textbf{自适应控制}:通过对于未知模型参数进行\textbf{在线估计},然后将参数估计值带入并更新控制器的设计。
	
	【CAA云讲座预告】北京航空航天大学教授王薇:数据驱动自适应控制理论及应用
	
\end{frame}

\begin{frame}
	\frametitle{自适应与自整定}
	Self-tuning 自整定
	\begin{itemize}
		\item Continuous updating of controller parameters 控制器参数的持续更新
		\item Used for truly time-varying plants 用于真正时变的被控对象
	\end{itemize}
	
	Auto-tuning
	\begin{itemize}
		\item Once controller parameters near convergence, adaptation is stopped 一旦控制器参数接近收敛,自适应即停止
		\item Used for time invariant or very slowly varying processes 用于时不变或变化极慢的过程
		\item Used for periodic, usually on-demand tuning 用于周期性的、通常为按需的整定
	\end{itemize}
\end{frame}


\begin{frame}
	\frametitle{自适应与增益调度}
	
	\begin{columns}[t]
		\column{0.48\textwidth}
		\begin{figure}
			\includegraphics[height=6cm]{figure/gain_schedualing}
		\end{figure}
		
		\column{0.48\textwidth}
		自适应控制是 “以变应变的智能调节”,增益调度控制是 “按预设规则的被动切换”。前者适合未知时变场景,后者适合已知工况的规律变化场景。
		
		\begin {itemize}\item 时变动态系统的控制\item 若动态特性随运行工况以已知、可预测的方式变化,采用 \textbf {增益调度}\item 若使用固定控制器无法在 \textbf {鲁棒性} 和 \textbf {性能} 之间达成满意的折中,当且仅当此时,才应使用自适应控制。\end {itemize}
		
		\textbf {使用满足技术要求的最简单技术}
	\end{columns}
	
\end{frame}


\subsection{为什么使用自适应控制}


\begin{frame}{传统控制的局限性:为什么需要新方案?}
	\begin{block}{工业系统的普遍挑战:不确定性}
		实际工程中,系统往往存在难以预先建模的\textbf{内外部不确定性},导致传统固定参数控制器失效:
		\begin{itemize}
			\item \textbf{内部不确定性}:
			\begin{itemize}
				\item 参数时变(如化工反应釜催化剂活性衰减、电机老化导致电阻变化)
				\item 未建模动态(如高层建筑风振、柔性机械臂弹性形变)
			\end{itemize}
			\item \textbf{外部不确定性}:
			\begin{itemize}
				\item 环境干扰(如无人机遭遇阵风、电网电压波动)
				\item 任务切换(如机器人负载突变、飞行器飞行模式转换)
			\end{itemize}
		\end{itemize}
	\end{block}
	
	\begin{exampleblock}{传统控制的“困境”}
		固定参数控制器仅针对“理想工况”设计,当系统特性偏离预设模型时,会出现:
		\begin{center}
			控制精度下降 $\to$ 系统稳定性恶化 $\to$ 甚至引发安全事故
		\end{center}
	\end{exampleblock}
\end{frame}




\begin{frame}{自适应控制的核心价值:解决传统方法无法应对的场景}
	\begin{columns}[t]
		\column{0.5\textwidth}
		\begin{block}{传统控制的“被动适应”局限}
			\begin{itemize}
				\item 增益调度控制:仅能应对\textbf{已知规律的工况变化}(如预设飞机马赫数-参数表),无法处理突发扰动
				\item 鲁棒控制:通过“保守设计”保证稳定性,但牺牲了\textbf{动态性能}(如响应速度变慢)
				\item 最优控制:依赖精确模型,模型失配时会导致\textbf{性能剧烈下降}
			\end{itemize}
		\end{block}
		
		\column{0.5\textwidth}
		\begin{block}{自适应控制的“主动调节”优势}
			针对不确定性,自适应控制通过“在线辨识+实时调参”实现:
			\begin{itemize}
				\item 无需预知参数变化规律,自动跟踪\textbf{未知时变特性}
				\item 在保证稳定性的同时,维持\textbf{高性能指标}(如最小跟踪误差)
				\item 适用于“模型难以精确建立”的复杂系统(如生物反应器、自动驾驶)
			\end{itemize}
		\end{block}
	\end{columns}
	
	\vspace{1em}
	\begin{center}
		\textbf{核心差异:从“以不变应万变”到“以变应变”}
	\end{center}
\end{frame}




\begin{frame}{实际应用中的迫切需求:自适应控制的不可替代性}
	\begin{block}{典型场景:为什么必须用自适应控制?}
		\begin{enumerate}
			\item \textbf{航空航天领域}:
			飞行器在宽空域(高度0-30km)、宽速域(马赫数0.3-5)飞行时,气动参数变化可达\textbf{10倍以上},传统控制器无法全覆盖,需自适应控制保证姿态稳定(如NASA X-43高超声速飞行器)。
			
			\item \textbf{工业制造领域}:
			13万吨级造纸机换产时,纸张定量、湿度等参数突变,自适应控制可将调整时间从传统方法的2小时缩短至\textbf{15分钟},降低废品率30%(某造纸厂实际案例)。
			
			\item \textbf{机器人与智能装备}:
			服务机器人抓取不同重量物体(0.1-5kg)时,动力学参数剧变,自适应控制可实现无超调的力/位置跟踪,避免传统控制的震荡问题。
		\end{enumerate}
	\end{block}
	
	\begin{alertblock}{总结:自适应控制的核心目标}
		在“模型不精确、参数时变、干扰未知”的场景下,实现:
		\begin{center}
			稳定运行 $\to$ 性能优化 $\to$ 降低人工干预 $\to$ 扩展系统适用范围
		\end{center}
	\end{alertblock}
\end{frame}


\begin{frame}
	\frametitle{没有自适应控制的时候}
	\begin{figure}
		\includegraphics[height=6cm]{figure/PID_pure}
	\end{figure}
\end{frame}


\begin{frame}
	\frametitle{有了自适应控制的时候}
	\begin{figure}
		\includegraphics[height=6cm]{figure/PID_adaptive}
	\end{figure}
\end{frame}

% \subsection{自适应控制的发展历史}

% \begin{frame}{自适应控制发展历程(一):理论奠基期(1950s-1960)}
% 	\begin{itemize}
% 		\item Mid 1950's: Flight control systems (eventually solved by gain scheduling)
% 		\item 1957: Bellman develops dynamic programming
% 		\item 1958: Kalman develops the self-optimizing controller \\
% 		``which adjusts itself automatically to control an arbitrary dynamic process''
% 		\item 1960: Feldbaum develops the dual controller in which the control action serves a dual purpose as it is ``directing as well as investigating''
% 	\end{itemize}
% \end{frame}

% \begin{frame}{自适应控制发展历程(二):技术成型与深化期(1960s-1990s)}
% 	\begin{itemize}
% 		\item Mid 60's-early 70's: Model reference adaptive systems
% 		\item Late 60's-early 70's: System identification approach with recursive least-squares
% 		\item Early 1980's: Convergence and stability analysis
% 		\item Mid 1980's: Robustness analysis
% 		\item 1990's: Multimodel adaptive control
% 		\item 1990's: Iterative control
% 	\end{itemize}
% \end{frame}
\section{Barbalat 引理}

\begin{frame}
  \frametitle{Barbalat 引理}
  \framesubtitle{函数及其导数的渐进性质}
  $\dot{f}\to 0 $
  $\nRightarrow$
  $f$ 收敛

  函数的倒数趋于零$\dot{f}\to 0 $并不等价于函数收敛$\exists C\in\mathbb{R},\lim_{t\to\infty}f(t)=C$.

  \begin{block}{example}
    \begin{enumerate}
      \item 有界函数:$f(t)=\sin(\ln t)$,$\dot{f}(t)=\frac{\cos(\ln t)}{t}$
      \item 无界函数:$f(t)=\sqrt{t} \sin(\ln t)$,$\dot{f}(t)=\frac{\sin(\ln t)}{2 \sqrt{t}}+ \frac{\cos (\ln t)}{\sqrt{t}}$
    \end{enumerate}
    这两个函数的导数都趋于零,但是函数不收敛 
  \end{block}
\end{frame}

\begin{frame}
  \frametitle{Barbalat 引理}
  \framesubtitle{函数及其导数的渐进性质}

  $f$ 收敛
  $\nRightarrow$
  $\dot{f}\to 0 $

  函数收敛并不意味着函数倒数趋于零$\dot{f}\to 0$.

  \begin{block}{Example}
    $f(t)=e^{-t}\sin (e^{2t})$收敛到零,但是其导数
    \[
      \dot{f}=-e^{-t}\sin(e^{2t}) + 2 e^{t} \cos(e^{2t})
    \]
    是一个有界函数。
  \end{block}

  \begin{block}{Tip: 与Lyapunov的联系}
    这个现象说明:
    如果一个函数$f$有下界并且为非增函数$\dot{f}\leq 0$,
    那么这个函数收敛到一个极限。
    \textbf{但是这个函数的导数不一定收敛到零}
  \end{block}
\end{frame}


\begin{frame}{Barbalat 引理}

  我们已知:若 $\lim_{t \to \infty} f(t) = 0$,则 $\lim_{t \to \infty} \intt{t}{f(\tau)} = C$(积分收敛,$C$ 为常数)。
  
  \vspace{0.5cm}
  \textbf{反过来成立吗?}  
  即:若 $\intt{\infty}{f(\tau)}$ 收敛(积分趋于常数),能否推出 $\lim_{t \to \infty} f(t) = 0$?

  \vspace{0.8cm}
  \begin{block}{反例:积分收敛,但函数不趋于零}
    回顾之前的三类函数(以矩形脉冲为例):
    \[ f_n(t) = n \cdot \chi_{[0, 1/n^2]}(t) \]
    ($\chi$ 为指示函数,$t \in [0,1/n^2]$ 时 $f_n(t)=n$,否则为0)
    
    - 积分特性:$\int_0^\infty f_n(t)dt = n \cdot \frac{1}{n^2} = \frac{1}{n} \to 0$(积分收敛);
    - 函数特性:$\max_{t \in \R} f_n(t) = n \to \infty$(函数不趋于零,反而发散)。
  \end{block}
\end{frame}

% ---------------------- 第3页:Barbalat 引理 —— 准确陈述 ----------------------
\begin{frame}{Barbalat 引理:条件与结论}
  Barbalat 引理是解决“积分收敛 $\Rightarrow$ 函数收敛”的关键工具,核心是通过**额外条件**弥补上述矛盾。

  \vspace{0.5cm}
  \begin{theorem}[经典 Barbalat 引理]
    设函数 $f(t): [0, \infty) \to \R$ 满足以下两个条件:
    \begin{enumerate}
      \item $f(t)$ 在 $[0, \infty)$ 上\textbf{连续};
      \item 无穷积分 $\intlim{0}{f(\tau)}$ \textbf{收敛}(即 $\lim_{t \to \infty} \intt{t}{f(\tau)} = C$,$C$ 为常数);
      \item $f'(t)$ 在 $[0, \infty)$ 上\textbf{一致连续}(或 $f(t)$ 本身一致连续)。
    \end{enumerate}
    则必有:$\boxed{\lim_{t \to \infty} f(t) = 0}$。
  \end{theorem}

  \vspace{0.3cm}
  \begin{remark}
    - 条件3是核心:若缺少“一致连续”,则可能出现“积分收敛但函数不趋于零”(如前页反例);
    - 常见简化版:若 $f(t)$ 一致连续且 $\intlim{0}{f(\tau)}$ 收敛,则 $\lim_{t \to \infty} f(t) = 0$(更易验证)。
  \end{remark}
\end{frame}

% ---------------------- 第4页:条件分析 —— 为什么需要“一致连续”? ----------------------
\begin{frame}{条件的必要性:反例验证}
  为说明“一致连续”不可缺少,我们对比两类函数(结合之前的Python绘图结果):

  \begin{columns}[T] % 分栏布局
    \column{0.5\textwidth}
    \textbf{反例1:矩形脉冲函数}
    \[ f_n(t) = n \cdot \chi_{[0, 1/n^2]}(t) \]
    \begin{itemize}
      \item 积分:$\int_0^\infty f_n(t)dt = 1/n \to 0$(收敛);
      \item 连续性:$f_n(t)$ 在 $t=1/n^2$ 处间断(非一致连续);
      \item 结论:$f_n(t)$ 不趋于零(最大值 $n \to \infty$)。
    \end{itemize}
    \vspace{0.3cm}
    % \includegraphics[width=0.9\textwidth]{rectangle_pulse.png} % 插入之前Python画的脉冲图

    \column{0.5\textwidth}
    \textbf{反例2:高频振荡函数}
    \[ f_n(t) = \sqrt{n} \cdot \sin(nt) \]
    \begin{itemize}
      \item 积分:$\int_0^\infty f_n(t)dt = \frac{1 - \cos(nt)}{\sqrt{n}} \to 0$(收敛);
      \item 连续性:$f_n(t)$ 连续,但 $f_n'(t) = n^{3/2}\cos(nt)$ 无界(非一致连续);
      \item 结论:$f_n(t)$ 振幅 $\sqrt{n} \to \infty$(不趋于零)。
    \end{itemize}
    \vspace{0.3cm}
    % \includegraphics[width=0.9\textwidth]{oscillating_function.png} % 插入振荡函数图
  \end{columns}
\end{frame}

% ---------------------- 第5页:证明思路 —— 简化版反证法 ----------------------
\begin{frame}{证明思路:反证法(简化版)}
  假设结论不成立,即 $\lim_{t \to \infty} f(t) \neq 0$,推出矛盾。

  \begin{enumerate}
    \item \textbf{反设前提}:存在 $\varepsilon_0 > 0$ 和序列 $t_k \to \infty$($k \to \infty$),使得 $\abs{f(t_k)} \geq \varepsilon_0$。
    
    \item \textbf{利用一致连续性}:因 $f'(t)$ 一致连续,存在 $\delta > 0$,对任意 $t \in [t_k - \delta, t_k + \delta]$,有:
    \[ \abs{f'(t)} \leq \frac{\varepsilon_0}{2\delta} \]
    由拉格朗日中值定理:$\abs{f(t) - f(t_k)} \leq \abs{f'(\xi)} \cdot \delta \leq \frac{\varepsilon_0}{2}$,故 $\abs{f(t)} \geq \frac{\varepsilon_0}{2}$。

    \item \textbf{积分矛盾}:对区间 $[t_k - \delta, t_k + \delta]$,积分满足:
    \[ \abs{\int_{t_k - \delta}^{t_k + \delta} f(t)dt} \geq \frac{\varepsilon_0}{2} \cdot 2\delta = \varepsilon_0 \delta > 0 \]
    但 $\intlim{0}{f(t)}$ 收敛,当 $k \to \infty$ 时,任意区间的积分应趋于0,矛盾。

    \item \textbf{结论}:反设不成立,故 $\lim_{t \to \infty} f(t) = 0$。
  \end{enumerate}
\end{frame}

% ---------------------- 第6页:核心应用 —— 控制理论中的稳定性分析 ----------------------
\begin{frame}{Barbalat 引理的核心应用:控制理论}
  Barbalat 引理是**Lyapunov稳定性分析**的重要补充,尤其用于证明“跟踪误差收敛”。

  \vspace{0.5cm}
  \begin{example}[自适应控制中的误差收敛]
    考虑跟踪系统:设跟踪误差为 $e(t)$,设计Lyapunov函数 $V(t)$ 满足:
    \[ \dot{V}(t) \leq -k e^2(t) \quad (k > 0 \text{ 为常数}) \]
    需证明 $\lim_{t \to \infty} e(t) = 0$,步骤如下:
    \begin{enumerate}
      \item 积分 $\dot{V}(t)$:$\int_0^t \dot{V}(\tau)d\tau = V(t) - V(0) \leq -k \int_0^t e^2(\tau)d\tau$;
      \item 因 $V(t) \geq 0$(Lyapunov函数性质),故 $\int_0^\infty e^2(\tau)d\tau \leq \frac{V(0)}{k}$(积分收敛);
      \item 验证一致连续:若系统输入有界,则 $\dot{e}(t)$ 有界 $\Rightarrow e(t)$ 一致连续;
      \item 应用Barbalat引理:$\int_0^\infty e^2(\tau)d\tau$ 收敛 + $e^2(t)$ 一致连续 $\Rightarrow \lim_{t \to \infty} e^2(t) = 0 \Rightarrow \lim_{t \to \infty} e(t) = 0$。
    \end{enumerate}
  \end{example}

  \vspace{0.3cm}
  \begin{block}{其他应用}
    - 信号处理:判断滤波后信号的收敛性;
    - 微分方程:证明解的渐近行为(如扰动系统的稳态误差)。
  \end{block}
\end{frame}

% ---------------------- 第7页:总结 ----------------------
\begin{frame}{总结:Barbalat 引理的核心价值}
  \begin{center}
    \begin{tabular}{|c|c|}
      \hline
      \textbf{核心问题} & 积分收敛 $\nRightarrow$ 函数收敛(需额外条件) \\
      \hline
      \textbf{Barbalat条件} & 1. 函数连续;2. 积分收敛;3. 导数(或函数)一致连续 \\
      \hline
      \textbf{关键结论} & 满足条件 $\Rightarrow \lim_{t \to \infty} f(t) = 0$ \\
      \hline
      \textbf{核心应用} & 控制理论(Lyapunov稳定性+误差收敛) \\
      \hline
    \end{tabular}
  \end{center}

  \vspace{1cm}
  \centering
  \textit{Barbalat引理的本质:用“一致连续性”约束函数的“波动幅度”,避免积分收敛但函数发散的情况。}
\end{frame}

\begin{frame}{Barbalat's lemma}
    
\end{frame}

\begin{frame}{核心思想:积分与函数的“分离性”}
  \begin{block}{关键区别}
    积分的“累积效应” ≠ 函数的“点态取值”:
    \begin{itemize}
      \item 积分趋于零:只需控制“函数高度 × 非零区间宽度”的乘积趋于零;
      \item 函数趋于零:需函数在“所有点”(或绝大多数点)的取值均趋于零。
    \end{itemize}
  \end{block}

  \vspace{0.5cm}
  \begin{center}
    \textbf{核心目标:构造满足“积分→0,但函数不→0”的函数}
  \end{center}
\end{frame}

% ---------------------- 第4页:例1 矩形脉冲函数 ----------------------
\section{场景1:固定区间上的函数序列}
\begin{frame}{例1:矩形脉冲函数(固定区间$[0,1]$)}
  \begin{columns}
    % 左列:函数定义与直观
    \column{0.45\textwidth}
    \begin{block}{1. 函数定义}
      对正整数 $n \to \infty$,定义:
      \[
      f_n(x) = 
      \begin{cases} 
      n & ,\ x \in \left[0, \frac{1}{n^2}\right] \\
      0 & ,\ x \in \left(\frac{1}{n^2}, 1\right]
      \end{cases}
      \]
      直观:“窄脉冲”——宽度$\frac{1}{n^2} \to 0$,高度$n \to \infty$。
    \end{block}

    % 右列:积分计算与趋势
    \column{0.55\textwidth}
    \begin{block}{2. 积分趋于零的证明}
      计算$[0,1]$上的定积分:
      \[
      \int_0^1 f_n(x) dx = \int_0^{\frac{1}{n^2}} n \, dx = n \times \frac{1}{n^2} = \frac{1}{n}
      \]
      当 $n \to \infty$ 时,$\frac{1}{n} \to 0$,故积分趋于零。
    \end{block}

    \begin{block}{3. 函数不趋于零的原因}
      取随$n$变化的点 $x_n = \frac{1}{2n^2}$,则:
      \[
      f_n(x_n) = n \to \infty \quad (n \to \infty)
      \]
      函数在“移动的窄区间”上取值无界,故不趋于零。
    \end{block}
  \end{columns}
\end{frame}

% ---------------------- 第5页:例2 高频震荡函数 ----------------------
\begin{frame}{例2:高频震荡函数(固定区间$[0,1]$)}
  \begin{columns}
    % 左列:函数定义与直观
    \column{0.45\textwidth}
    \begin{block}{1. 函数定义}
      对正整数 $n \to \infty$,定义:
      \[
      f_n(x) = \sqrt{n} \cdot \sin(nx) \quad (x \in [0,1])
      \]
      直观:高频震荡的正弦函数——振幅$\sqrt{n} \to \infty$,频率$n \to \infty$。
    \end{block}

    % 右列:积分计算与趋势
    \column{0.55\textwidth}
    \begin{block}{2. 积分趋于零的证明}
      计算定积分:
      \[
      \int_0^1 \sqrt{n}\sin(nx)dx = \sqrt{n} \cdot \left. \frac{-\cos(nx)}{n} \right|_0^1 = \frac{1 - \cos(n)}{\sqrt{n}}
      \]
      因 $|\cos(n)| \leq 1$,故 $|1 - \cos(n)| \leq 2$,进而:
      \[
      \left| \frac{1 - \cos(n)}{\sqrt{n}} \right| \leq \frac{2}{\sqrt{n}} \to 0 \quad (n \to \infty)
      \]
    \end{block}

    \begin{block}{3. 函数不趋于零的原因}
      取 $x_n = \frac{\pi}{2n}$,则 $\sin(nx_n) = 1$,故:
      \[
      f_n(x_n) = \sqrt{n} \to \infty \quad (n \to \infty)
      \]
      振幅无界,函数不趋于零。
    \end{block}
  \end{columns}
\end{frame}

% ---------------------- 第6页:例3 无穷区间函数 ----------------------
\section{场景2:无穷区间上的函数}
\begin{frame}{例3:傅里叶积分型函数 $f(x) = \sin(x^2)$(区间$[1,+\infty)$)}
  \begin{block}{1. 函数定义与直观}
    定义:$f(x) = \sin(x^2)$($x \in [1,+\infty)$),特点是“频率递增的震荡函数”:
    - $x^2$ 增速随$x$增大而加快,导致$\sin(x^2)$震荡越来越频繁;
    - 振幅始终为1(有界但不趋于零)。
  \end{block}

  \begin{block}{2. 局部积分趋于零($A \to +\infty$)}
    计算局部积分 $\int_A^{A+1} \sin(x^2)dx$,作变量代换 $t = x^2$($dx = \frac{dt}{2\sqrt{t}}$):
    \[
    \int_A^{A+1} \sin(x^2)dx = \int_{A^2}^{(A+1)^2} \frac{\sin t}{2\sqrt{t}} dt
    \]
    由积分中值定理,存在 $\xi \in [A^2, (A+1)^2]$ 使:
    \[
    \text{积分} = \frac{\sin \xi}{2\sqrt{\xi}} \cdot (2A+1)
    \]
    因 $\sqrt{\xi} \geq A$,故 $\frac{2A+1}{2\sqrt{\xi}} \leq \frac{2A+1}{2A} \to 1$,且 $|\sin \xi| \leq 1$,因此:
    \[
    |\text{积分}| \leq \frac{2A+1}{2A} \cdot \frac{1}{\sqrt{\xi}} \to 0
    \]
  \end{block}

  \begin{block}{3. 函数不趋于零的原因}
    取点列 $x_k = \sqrt{\frac{\pi}{2} + 2k\pi}$($k \to +\infty$),则:
    \[
    f(x_k) = \sin\left(\frac{\pi}{2} + 2k\pi\right) = 1 \not\to 0
    \]
    函数始终在$[-1,1]$震荡,无极限。
  \end{block}
\end{frame}

% ---------------------- 第7页:核心总结 ----------------------
\section{核心总结}
\begin{frame}{核心总结}
  \begin{block}{构造“积分→0但函数不→0”的两种核心机制}
    \begin{enumerate}
      \item \textbf{脉冲机制}:
        - 函数仅在“宽度→0”的区间上非零,高度→∞,但“高度×宽度→0”(如矩形脉冲);
        - 关键:用“窄区间”抵消“高取值”的累积效应。

      \item \textbf{震荡抵消机制}:
        - 函数高频震荡,正负面积相互抵消,积分→0,但振幅不→0(如$\sqrt{n}\sin(nx)$、$\sin(x^2)$);
        - 关键:用“高频震荡”实现面积抵消。
    \end{enumerate}
  \end{block}

  \vspace{0.5cm}
  \begin{alertblock}{重要结论}
    \centering
    \textbf{“积分趋于零”不能推出“函数趋于零”} \\
    积分收敛反映“累积效应→0”,而非“点态取值→0”。
  \end{alertblock}
\end{frame}

\section{自适应控制}

\subsection{未知参数自适应控制}

\begin{frame}
    \frametitle{参数自适应}

    考虑一维镇定问题($x,w,u\in[0,\infty)\mapsto\mathbb{R}$):
    \begin{equation}
        \dot{x}=w(t) \mu +u
    \end{equation}
    其中$\mu\in\mathbb{R}$是未知恒定参数,设计控制器为:
    \begin{equation}
        u=-x-w(t)\hat{\mu},
        \quad
        \dot{\hat{\mu}}=-xw(t)
    \end{equation}

    于是闭环的自适应控制系统可以写作:
    \[
    \begin{cases}
    \dot{e} = -e + \tilde{\mu} w(t) \\
    \dot{\tilde{\mu}} = -e w(t)
    \end{cases}
    \]

    其中,\( e=x \) 是跟踪误差,\( \tilde{\mu}=\hat{\mu}-\mu \) 是参数误差,\( w(t) \) 是一个有界连续函数。
\end{frame}

\begin{frame}
考虑下方有界函数:
\[
\begin{split}
V &= e^2 + \tilde{\mu}^2 \\
\dot{V} &= 2e(-e + \tilde{\mu} w) + 2\tilde{\mu}(-ew(t)) = -2e^2 \leq 0
\end{split}
\]
所以 \( V(t) \leq V(0) \),因此 \( e \) 和 \( \tilde{\mu} \) 是有界的。

由于系统的$\dot{V}\leq 0$,无法直接保证渐进稳定。
由于动力学是非自治的,\textbf{不变集定理}不能用来推断 \( e \) 的收敛性。
因此,需要使用基于Barbalat's 引理的LaSalle-Yoshizawa定理。

通过$\ddot{V}$的有界性,检查 \( \dot{V} \) 的一致连续性。
\[
\ddot{V} = -4e(-e + \tilde{\mu} w(t))
\]

\( \ddot{V} \) 是有界的,因为根据假设 \( w \) 是有界的,且 \( e \) 和 \( \tilde{\mu} \) 已被证明是有界的 \( \rightsquigarrow \dot{V} \) 是一致连续的。

应用 Barbalat 引理:当 \( \dot{V} = 0 \) 时,\( t \to \infty \) 时 \( e \to 0 \)。

\textbf{重要提示}:尽管 \( e \to 0 \),但整个系统不是渐近稳定(a.s.)的,因为仅证明了 \( \tilde{\mu} \) 是有界的。
\end{frame}


\begin{frame}
    \frametitle{什么时候观测误差也渐进收敛?}

    通过分析已知$e\to 0$, $\dot{e}\to0$(Barbalat's lemma).

    \[
    \begin{cases}
    \dot{e} = -e + \tilde{\mu} w(t) \\
    \dot{\tilde{\mu}} = -e w(t)
    \end{cases}
    \]
    第一行:$\tilde{\mu} w(t)=\dot{e}+e\to 0$.
    第二行:$\dot{\tilde{\mu}} = -e w(t)\to 0$.

    \begin{block}{$w(t)$需要满足什么条件能够得到$\tilde{\mu}(t)\to 0$}
        \footnote{示例来源于\cite[Example 2.7]{Chen2015}}

    如果$w(t)$是一个非零的常数,可得$\lim_{t\to\infty}\tilde{\mu}(t)=0$.
    
    如果$w(t)=[\sin(\omega t),\sin(\omega t)]^T$的时候,那么显然存在$\tilde{\mu}(t)=[1 -1]^T$的解,不满足$\lim_{t\to\infty}\tilde{\mu}(t)=0$.
        
    \end{block}
\end{frame}
\subsection{PE 条件}

\begin{frame}
    \frametitle{PE条件:一个充分条件}

    通过分析已知$e\to 0$, $\dot{e}\to0$(Barbalat's lemma)
    可知$\tilde{\mu} w(t)\to 0$.

    \[
    \begin{cases}
    \dot{e} = -e + \tilde{\mu} w(t) \\
    \dot{\tilde{\mu}} = -e w(t)
    \end{cases}
    \]

    \begin{lemma}
        Lemma 2.4 \cite{Chen2015} Sec 2.3 p 29.
        考虑一个连续可微的函数$g:[0,\infty)\mapsto \mathbb{R}^n$ 和一个有界分段连续的函数$f:[0,\infty)\mapsto \mathbb{R}^n$满足
        \[\lim_{t\to\infty} g^T(t)f(t)=0\]
        当满足以下两个条件的时候:
        \begin{enumerate}
            \item $\lim_{t\to\infty} \dot{g}(t)=0$
            \item $f(t)$ is PE.
        \end{enumerate}
        有 $\lim_{t\to\infty}=0$.
    \end{lemma}
\end{frame}
\subsection{自适应估计}


\begin{frame}
    自适应控制与系统辨识中最基本的问题是,从标量输出 \( y \in \mathbb{R} \) 估计参数向量 \( \theta \in \mathbb{R}^n \),它们满足线性回归关系
\[
y(t) = \phi(t)^\top \theta.
\]

若要得唯一解,则至少有 \( n \) 个时刻方阵 \( [\phi(t_1), \dots, \phi(t_n)] \) 满秩,这是所谓的可辨识性。在控制问题中我们倾向于持续地在线估计,这样既可保证鲁棒性,又可以在线跟踪缓慢的参数变化。这就要把可辨识性,即 \( [\phi(t_1), \dots, \phi(t_n)] \) 满秩,大体上“一直保持下去”,以上即持续激励的基本出发点。对于以上线性回归问题,最基本的梯度下降法,可得如下线性时变误差系统
\[
\dot{\hat{\theta}}= -\gamma \phi(t) y(t)
\]\[
\dot{\tilde{\theta}} = -\gamma \phi(t) \phi(t)^\top \tilde{\theta}.
\]
\end{frame}

\begin{frame}
    \begin{itemize}
    \item 从动力系统稳定性角度来看,以上系统原点指数稳定 \( \Leftrightarrow \phi(t) \) 是持续激励的,即存在常数 \( T, k \) 使得在任意时刻 \( t \) 都有
    \[
    \int_{t}^{t+T} \phi(\tau) \phi(\tau)^\top d\tau \geq k I_{n \times n}.
    \]
    \item 从几何的角度来看,持续激励是说,在每个时间段 \( [t, t+T] \) 内,\( \phi(t) \) 旋转过方向所张成的空间要能够覆盖整个 \( \mathbb{R}^n \) 空间。显然对于任意时刻 \( \text{rank}\{\phi(t) \phi(t)^\top\} = 1 \),只有在每个时间段 \( [t, t+T] \) 内,\( \phi(t) \) 旋转过 \( n \) 个线性无关的“方向”,才能让以上持续激励条件成立。
\end{itemize}
\end{frame}

\begin{frame}
    \begin{itemize}
    \item 从平均理论的角度来说,持续激励是让(时变的)系统矩阵始终为正定。误差系统的平均动力学“看上去”是(ODE教材里有严格的平均分析)
    \[
    \dot{\bar{\tilde{\theta}}} = -\gamma \bar{A}(t) \bar{\tilde{\theta}}, \quad \bar{A} := \frac{1}{T} \int_{t}^{t+T} \phi(\tau)\phi(\tau)^\top d\tau
    \]
    其中 \( \bar{\tilde{\theta}} \) 是 \( \tilde{\theta} \) 的平均变量。平均系统矩阵 \( \bar{A}(t) \) 在持续激励条件下为正定,状态将收敛至平衡点。

    \item 从频域角度来看,持续激励条件要求系统输入的频谱丰富程度大于等于 \( n \)。最简单的情况就是回归矩阵 \( \phi(t) \) 是由单入单出线性时不变系统产生,那么该LTI系统输入含有不少于 \( n \) 个频谱时,即输入 \( u \) 可以分解为
    \[
    u = \sum_{i=1}^{m} \sin(\omega_i t + \psi_i)
    \]
    其中 \( m \geq n/2 \)(每个正弦信号提供2个频谱),那么回归矩阵 \( \phi(t) \) 是持续激励的。这也是充分必要条件。
\end{itemize}
\end{frame}
\subsection{未知控制输入方向的自适应控制器设计}

\begin{frame}{未知控制输入方向的自适应控制器设计}
    考察最最简单的未知增益系统 \( \dot{x} = gu \),这里 \( x \in \mathbb{R} \) 为系统状态,\( u \in \mathbb{R} \) 为系统控制输入,\( g \in \mathbb{R}, g \neq 0 \) 为系统的未知的增益。我们可以设计如下的控制律来实现系统的镇定 \( x \to 0 \):

    \[
    u = \mathcal{N}(z)x, \quad \dot{z} = x^2
    \]

    这里的 \( \mathcal{N}(z) \) 为一个 Nussbaum 函数,具体定义请参看\footfullcite[Section 6.3]{Chen2015}。

\end{frame}


\begin{frame}
    
\textbf{定义 6.2} 一个连续函数 \( v: \mathbb{R} \mapsto \mathbb{R} \) 被称为 \( \mathcal{N} \) 类函数,记为 \( v \in \mathcal{N} \),如果
\[
\liminf_{k \to \infty} \frac{k - \int_{0}^{k} v^{-}(s)ds}{\int_{0}^{k} v^{+}(s)ds} = 0, \tag{6.56}
\]
\[
\liminf_{k \to \infty} \frac{k + \int_{0}^{k} v^{+}(s)ds}{-\int_{0}^{k} v^{-}(s)ds} = 0. \tag{6.57}
\]

(其中 \( v^{+}(s) \) 通常表示 \( v(s) \) 的正部,即 \( v^{+}(s)=\max\{v(s),0\} \);\( v^{-}(s) \) 表示 \( v(s) \) 的负部,即 \( v^{-}(s)=\max\{-v(s),0\} \),用于将函数分解为正、负两部分来分析。)
\end{frame}

\begin{frame}
    \textbf{Lemma 6.3} 如果 \( v \in \mathcal{N} \),那么
\[
\limsup_{k \to \infty} \frac{1}{k} \int_{0}^{k} v(s) ds = +\infty, \tag{6.58}
\]
\[
\liminf_{k \to \infty} \frac{1}{k} \int_{0}^{k} v(s) ds = -\infty. \tag{6.59}
\]

\begin{center}
    \includegraphics[width=0.9\linewidth]{python/out/nussbaum.pdf}
\end{center}
\end{frame}

\begin{frame}{仿真效果}
    $g=1$
    \begin{figure}
        \centering\includegraphics[width=0.45\linewidth]{python/out/n1_x.pdf}
        \includegraphics[width=0.45\linewidth]{python/out/n1_z.pdf}
        \includegraphics[width=0.45\linewidth]{python/out/n1_u.pdf}
        \includegraphics[width=0.45\linewidth]{python/out/n1_n.pdf}
    \end{figure}
\end{frame}

\begin{frame}{仿真效果}
    $g=-1$
    \begin{figure}
        \centering\includegraphics[width=0.45\linewidth]{python/out/n2_x.pdf}
        \includegraphics[width=0.45\linewidth]{python/out/n2_z.pdf}
        \includegraphics[width=0.45\linewidth]{python/out/n2_u.pdf}
        \includegraphics[width=0.45\linewidth]{python/out/n2_n.pdf}
    \end{figure}
\end{frame}
\section{Duffing 系统的自适应控制}

\begin{frame}
    \frametitle{Duffing 系统}
    
\end{frame}

\section{总结}

\begin{frame}
  \frametitle{总结}
  \begin{enumerate}
    \item 一个函数收敛到零,当函数两阶导有界时(充分),其导数收敛到零。(Barbalat's 引理)
    \item 对于非自治系统,除了需要验证$V\geq 0$, $\dot{V}\leq 0$还需要验证$\ddot{V}$有界(LaSalle-Yoshizawa 定理)
    \item 在将输出镇定为零的控制问题中,线性系统通常也会将所有状态镇定为零,而自适应控制中不一定所有的状态最后都收敛到了零。(PE条件)
    \item 关于状态发散的信号在未知控制方向的自适应控制中有着广泛的应用(Nussbaum)。
  \end{enumerate}
  
\end{frame}

% 显示参考文献列表
\begin{frame}[allowframebreaks]{参考文献}
  \printbibliography
  % allowframebreaks选项允许参考文献跨页显示
\end{frame}

\begin{frame}
  \frametitle{那么,代价是什么呢?}
    未知参数自适应控制需要PE条件,难以保证。

     \[
    \begin{cases}
    \dot{e} = -e + \tilde{\mu} w(t) \\
    \dot{\tilde{\mu}} = -e w(t)
    \end{cases}
    \]

    未知控制方向自适应控制的参数通常会大于实际所需,特别是当存在一定扰动的时候,误差累积会影响控制效果。

    \[
    \dot{x} = g\mathcal{N}(z)x, \quad \dot{z} = x^2
    \]

    \centering
    \includegraphics[width=0.45\linewidth]{python/out/n3delta_x.pdf}
    \includegraphics[width=0.45\linewidth]{python/out/n3delta_n.pdf}
\end{frame}

% 感谢页(核心内容)
\begin{frame}[plain] % plain 去除页眉页脚,更简洁
    \centering
    
    % 大标题:感谢聆听
    {\Huge \textbf{感谢}}\\[2em]
  
  
\end{frame}

\end{document}