\subsection{未知控制输入方向的自适应控制器设计}

\begin{frame}{未知控制输入方向的自适应控制器设计}
    考察最最简单的未知增益系统 \( \dot{x} = gu \),这里 \( x \in \mathbb{R} \) 为系统状态,\( u \in \mathbb{R} \) 为系统控制输入,\( g \in \mathbb{R}, g \neq 0 \) 为系统的未知的增益。我们可以设计如下的控制律来实现系统的镇定 \( x \to 0 \):

    \[
    u = \mathcal{N}(z)x, \quad \dot{z} = x^2
    \]

    这里的 \( \mathcal{N}(z) \) 为一个 Nussbaum 函数,具体定义请参看\footfullcite[Section 6.3]{Chen2015}。

\end{frame}


\begin{frame}
    
\textbf{定义 6.2} 一个连续函数 \( v: \mathbb{R} \mapsto \mathbb{R} \) 被称为 \( \mathcal{N} \) 类函数,记为 \( v \in \mathcal{N} \),如果
\[
\liminf_{k \to \infty} \frac{k - \int_{0}^{k} v^{-}(s)ds}{\int_{0}^{k} v^{+}(s)ds} = 0, \tag{6.56}
\]
\[
\liminf_{k \to \infty} \frac{k + \int_{0}^{k} v^{+}(s)ds}{-\int_{0}^{k} v^{-}(s)ds} = 0. \tag{6.57}
\]

(其中 \( v^{+}(s) \) 通常表示 \( v(s) \) 的正部,即 \( v^{+}(s)=\max\{v(s),0\} \);\( v^{-}(s) \) 表示 \( v(s) \) 的负部,即 \( v^{-}(s)=\max\{-v(s),0\} \),用于将函数分解为正、负两部分来分析。)
\end{frame}

\begin{frame}
    \textbf{Lemma 6.3} 如果 \( v \in \mathcal{N} \),那么
\[
\limsup_{k \to \infty} \frac{1}{k} \int_{0}^{k} v(s) ds = +\infty, \tag{6.58}
\]
\[
\liminf_{k \to \infty} \frac{1}{k} \int_{0}^{k} v(s) ds = -\infty. \tag{6.59}
\]

\begin{center}
    \includegraphics[width=0.9\linewidth]{python/out/nussbaum.pdf}
\end{center}
\end{frame}

\begin{frame}{仿真效果}
    $g=1$
    \begin{figure}
        \centering\includegraphics[width=0.45\linewidth]{python/out/n1_x.pdf}
        \includegraphics[width=0.45\linewidth]{python/out/n1_z.pdf}
        \includegraphics[width=0.45\linewidth]{python/out/n1_u.pdf}
        \includegraphics[width=0.45\linewidth]{python/out/n1_n.pdf}
    \end{figure}
\end{frame}

\begin{frame}{仿真效果}
    $g=-1$
    \begin{figure}
        \centering\includegraphics[width=0.45\linewidth]{python/out/n2_x.pdf}
        \includegraphics[width=0.45\linewidth]{python/out/n2_z.pdf}
        \includegraphics[width=0.45\linewidth]{python/out/n2_u.pdf}
        \includegraphics[width=0.45\linewidth]{python/out/n2_n.pdf}
    \end{figure}
\end{frame}