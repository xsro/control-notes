\section{实例:Duffing 系统的自适应控制}


% 1. 模型建立
\begin{frame}{模型建立:受控Duffing方程}
  考虑受控Duffing方程,外部扰动重写为余弦/正弦组合:\footnote{Duffing系统 https://zhuanlan.zhihu.com/p/618829679}
  \[
  v(t) = \gamma \cos(\omega t + \phi) = \gamma_1 \cos(\omega t) + \gamma_2 \sin(\omega t)
  \]
  最终模型为:
  \begin{align}
  \dot{x}_1 &= x_2 \\
  \dot{x}_2 &= -\delta x_2 - \alpha x_1 - \beta x_1^3 + \gamma_1 \cos(\omega t) + \gamma_2 \sin(\omega t) + u. \tag{5.53}
  \end{align}

  \begin{block}{已知/未知信息}
    \begin{itemize}
      \item 未知:$\delta, \alpha, \beta, \gamma_1, \gamma_2$
      \item 已知:外部信号频率 $\omega$
    \end{itemize}
  \end{block}
\end{frame}


% 2. 状态变换与ISS分析
\begin{frame}{状态变换与ISS分析}
  定义新状态变量:
  \[
  \mathcal{X}_2 = x_2 + \rho_1 x_1 \quad (\rho_1 > 0)
  \]
  则 $x_1$ 的动力学满足:
  \[
  \dot{x}_1 = -\rho_1 x_1 + \mathcal{X}_2
  \]
  该式关于 $x_1$(状态)和 $\mathcal{X}_2$(输入)是\textbf{输入-状态稳定(ISS)}的。

  \begin{figure}
    \centering
    \includegraphics[height=4cm]{figure/mass-spring.png}
  \end{figure}
\end{frame}


% 3. 新动力学与控制器设计
\begin{frame}{$\mathcal{X}_2$ 动力学与自适应控制器}
  $\mathcal{X}_2$ 的动力学方程:
  \begin{align}
  \dot{\mathcal{X}}_2 = f_0^\top(x_1, x_2, t)\mu + u \tag{5.54}
  \end{align}
  其中向量定义:
  \[
  f_0(x_1, x_2, t) = \begin{bmatrix} x_2 \\ x_1 \\ x_1^3 \\ \cos(\omega t) \\ \sin(\omega t) \end{bmatrix}, \quad \mu = \begin{bmatrix} -\delta + \rho_1 \\ -\alpha \\ -\beta \\ \gamma_1 \\ \gamma_2 \end{bmatrix}
  \]

  根据定理5.4,自适应控制器为:
  \begin{align}
  u &= -\rho_2 \mathcal{X}_2 - f_0^\top(x_1, x_2, t)\hat{\mu} \ (\rho_2 > 0) \tag{5.55} \\
  \dot{\hat{\mu}} &= \Lambda \mathcal{X}_2 f_0(x_1, x_2, t)
  \end{align}
\end{frame}


% 4. 仿真结果与说明
\begin{frame}{仿真结果与方法局限性}
  \begin{block}{仿真参数}
    $\delta=0.03,\ \alpha=-1,\ \beta=1$,扰动 $v(t)=0.8\cos(0.2t+0.1)$
  \end{block}

  \begin{itemize}
    \item 系统状态渐近趋近于平衡点 $x=0$
    \item 观测器参数不能都收敛到真实值
  \end{itemize}

  \begin{alertblock}{方法局限性}
    本章方法无法处理\textbf{未知的 $\omega$}(因 $\cos(\omega t)/\sin(\omega t)$ 关于 $\omega$ 非线性),第9章的\textbf{内模方法}可解决此问题。
  \end{alertblock}
\end{frame}

\begin{frame}
    \frametitle{自适应控制效果}
    \centering
    \includegraphics[width=0.75\textwidth]{python/out/duffing-x.pdf}
\end{frame}

\begin{frame}
    \frametitle{参数收敛效果}
    \centering
    \includegraphics[width=0.75\textwidth]{python/out/duffing-mu.pdf}
\end{frame}