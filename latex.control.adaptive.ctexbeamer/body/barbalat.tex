\section{自适应控制相关定理}

\subsection{Barbalat 引理}

\begin{frame}
  \frametitle{Barbalat 引理}
  \framesubtitle{函数及其导数的渐进性质}
  $\dot{f}\to 0 $
  $\nRightarrow$
  $f$ 收敛

  函数的倒数趋于零$\dot{f}\to 0 $并不等价于函数收敛$\exists C\in\mathbb{R},\lim_{t\to\infty}f(t)=C$.

  \begin{block}{example}
    \begin{enumerate}
      \item 有界函数:$f_1(t)=\sin(\ln t)$,$\dot{f}_1(t)=\frac{\cos(\ln t)}{t}$
      \item 无界函数:$f_2(t)=\sqrt{t} \sin(\ln t)$,$\dot{f}_2(t)=\frac{\sin(\ln t)}{2 \sqrt{t}}+ \frac{\cos (\ln t)}{\sqrt{t}}$
    \end{enumerate}

    这两个函数的导数都趋于零,但是函数不收敛.

    \begin{figure}
      \centering
      \includegraphics[height=2.8cm]{python/out/sinlnt.pdf}
      \quad
      \includegraphics[height=2.8cm]{python/out/sqrttsinlnt.pdf}
    \end{figure}
  \end{block}
\end{frame}

\begin{frame}
  \frametitle{Barbalat 引理}
  \framesubtitle{函数及其导数的渐进性质}

  $f$ 收敛
  $\nRightarrow$
  $\dot{f}\to 0 $

  函数收敛并不意味着函数倒数趋于零$\dot{f}\to 0$.

  \begin{block}{Tip: 与Lyapunov的联系}
    这个现象说明:
    如果一个函数$f$有下界并且为非增函数$\dot{f}\leq 0$,
    那么这个函数收敛到一个极限。
    \textbf{但是这个函数的导数不一定收敛到零}
  \end{block}

  \begin{block}{Example 1}
    另一个例子是周期脉冲信号:

    \begin{figure}
      \centering
      \includegraphics[height=2.6cm]{python/out/barbalat_f4.pdf}
      \includegraphics[height=2.6cm]{python/out/barbalat_f4d.pdf}
    \end{figure}
  \end{block}
\end{frame}


\begin{frame}
  \begin{block}{Example 2}
    $f(t)=e^{-t}\sin (e^{2t})$收敛到零,但是其导数
    \[
      \dot{f}=-e^{-t}\sin(e^{2t}) + 2 e^{t} \cos(e^{2t})
    \]
    是一个有界函数。

    \begin{figure}
      \centering
      \includegraphics[height=2.6cm]{python/out/barbalat_f3.pdf}
      \includegraphics[height=2.6cm]{python/out/barbalat_f3d.pdf}
    \end{figure}
  \end{block}
\end{frame}

\begin{frame}
  \frametitle{Barbalat‘s 引理的基本形式}

  \begin{lemma}
  设 $x:[0, \infty) \to \mathbb{R}$ 为一阶连续可导,且当 $t \to \infty$ 时有极限,则如果 $\dot{x}(t), t \in [0, \infty)$ 一致连续,那么 $\lim\limits_{t \to \infty} \dot{x}(t) = 0$。
  \end{lemma} 

  如果 $\ddot{x}(t)$ 存在且有界,那么引理1中 $\dot{x}(t)$ 的一致连续性条件可用 $\ddot{x}(t)$ 的有界性来替代,从而得到如下形式的引理。

  \begin{lemma}
  设 $x:[0, \infty) \to \mathbb{R}$ 一阶连续可导,且当 $t \to \infty$ 时有极限,则如果 $\ddot{x}(t), t \in [0, \infty)$ 存在且有界,那么 $\lim\limits_{t \to \infty} \dot{x}(t) = 0$。
  \end{lemma}

  \begin{corollary}
    若 $x: [0, \infty) \to \mathbb{R}$ 一致连续,并且 $\lim\limits_{t \to \infty} \int_{0}^{t} x(\tau) \mathrm{d}\tau$ 存在且有界,那么 $\lim\limits_{t \to \infty} x(t) = 0$。
  \end{corollary}
\end{frame}

\subsection{LaSalle-Yoshizawa 定理(拉萨尔-吉泽定理)}

\begin{frame}
  \frametitle{回顾李亚普诺夫第二定理}

  考虑自治系统
  \[
  \dot{x} = f(x) \tag{1}
  \]
  
  \begin{theorem}
    \parencite[定理 3.3]{khalilNonlinearControl2015}
    设 $D \subset \mathbb{R}^n$ 是包含原点的一个区域,$f(x)$ 为定义在 $D$ 上的局部 Lipschitz 函数,且 $f(0)=0$。令 $V(x)$ 是定义在 $D$ 上的连续可微函数,且满足
    \[
    V(0) = 0, \text{以及}\ V(x) > 0,\quad x \in D \text{且} x \neq 0 \tag{3.7}
    \]
    并且\tB{\(
    \dot{V}(x) \leq 0,\quad x \in D \)}, 
    则原点是 $\dot{x}=f(x)$ 的稳定平衡点。
    
    如果进一步有\tB{  \(
    \dot{V}(x) < 0,\quad x \in D \text{且} x \neq 0
    \)}
    则原点是渐近稳定的。
    
    此外,如果 $D=\mathbb{R}^n$,式 (3.7) 与式 (3.9) 对所有 $x \neq 0$ 成立,且\textbf{径向无界}
    \(
    \|x\| \to \infty \Rightarrow V(x) \to \infty
    \)
    则原点是全局渐近稳定的。
  \end{theorem}
\end{frame}

\begin{frame}
  \frametitle{回顾不变性原理 1}
  考虑自治系统
\[
\dot{x} = f(x) \tag{1}
\]

\begin{theorem}
\parencite[定理 3.4]{khalilNonlinearControl2015,LaSalle1960}
设 $f(x)$ 是定义在域 $D\subset\mathbb{R}^n$ 上的局部 Lipschitz 函数, $\Omega\subset D$ 是一个紧集,并且关于 $\dot{x}=f(x)$ 是正向不变的。设 $V(x)$ 为定义在区域 $D$ 上的连续可微函数,在 $\Omega$ 内满足 $\dot{V}(x)\leq 0$。设 $E$ 是 $\Omega$ 内\tB{所有满足 $\dot{V}(x)=0$ 的 $x$ 组成的集合},记 $M$ 是 $E$ 内的最大不变集。那么当 $t\to\infty$ 时,始于 $\Omega$ 内的每个解都趋于 $M$。
\end{theorem}

\end{frame}

\begin{frame}
  \frametitle{回顾不变性原理 2}
  考虑自治系统
\[
\dot{x} = f(x) \tag{1}
\]

\begin{theorem}
\parencite[定理 3.5]{khalilNonlinearControl2015}
设 $D\subset\mathbb{R}^n$ 是包含原点的一个区域,$f(x)$ 是定义在 $D$ 上的局部 Lipschitz 函数,且 $f(0)=0$。设 $V(x)$ 是 $D$ 上连续可微的正定函数,\tB{且在 $D$ 内满足 $\dot{V}(x)\leq 0$}。设 \tB{$S=\{x\in D\mid \dot{V}(x)=0\}$},并假设除平凡解 $x(t)\equiv 0$ 之外,没有其他解一直保持在 $S$ 内,那么 $\dot{x}=f(x)$ 的原点是渐近稳定的平衡点。

进一步,设紧集 $\Gamma\subset D$ 关于 $\dot{x}=f(x)$ 正向不变,那么 $\Gamma$ 是吸引域的子集。最后,如果 $D=\mathbb{R}^n$,且 $V(x)$ 是径向无界的,那么原点是全局渐近稳定的。
\end{theorem}

\end{frame}


\subsection{Barbalat's 引理在李亚普诺夫分析中的应用}
\begin{frame}
  \frametitle{Barbalat's 引理在李亚普诺夫分析中的应用}

  考虑非自治系统
\[
\dot{x} = f(t,x) 
\]
其中 $x \in \mathbb{R}^n$,$f: D\times[0,\infty) \to \mathbb{R}^n$ 是局部 Lipschitz 函数,$D \subseteq \mathbb{R}^n$ 为定义域。
  
  \begin{theorem}[LaSalle-Yoshizawa 定理] 
    \parencite[Theorem 2.5]{Chen2015}
    如果一个标量函数$V(t,x)$ 满足如下条件:
    \begin{enumerate}
      \item $V(t,x)$ 有下界;
      \item $\dot{V}(t,x)$ 是半负定;
      \item $\dot{V}(t,x)$ 关于时间是一致连续的
    \end{enumerate}
    那么 有$\lim_{t\to\infty} \dot{V}(t,x)=0$
  \end{theorem} 
\end{frame}