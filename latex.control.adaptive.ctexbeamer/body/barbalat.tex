\section{Barbalat 引理}

\begin{frame}
  \frametitle{Barbalat 引理}
  \framesubtitle{函数及其导数的渐进性质}
  $\dot{f}\to 0 $
  $\nRightarrow$
  $f$ 收敛

  函数的倒数趋于零$\dot{f}\to 0 $并不等价于函数收敛$\exists C\in\mathbb{R},\lim_{t\to\infty}f(t)=C$.

  \begin{block}{example}
    \begin{enumerate}
      \item 有界函数:$f(t)=\sin(\ln t)$,$\dot{f}(t)=\frac{\cos(\ln t)}{t}$
      \item 无界函数:$f(t)=\sqrt{t} \sin(\ln t)$,$\dot{f}(t)=\frac{\sin(\ln t)}{2 \sqrt{t}}+ \frac{\cos (\ln t)}{\sqrt{t}}$
    \end{enumerate}
    这两个函数的导数都趋于零,但是函数不收敛 
  \end{block}
\end{frame}

\begin{frame}
  \frametitle{Barbalat 引理}
  \framesubtitle{函数及其导数的渐进性质}

  $f$ 收敛
  $\nRightarrow$
  $\dot{f}\to 0 $

  函数收敛并不意味着函数倒数趋于零$\dot{f}\to 0$.

  \begin{block}{Example}
    $f(t)=e^{-t}\sin (e^{2t})$收敛到零,但是其导数
    \[
      \dot{f}=-e^{-t}\sin(e^{2t}) + 2 e^{t} \cos(e^{2t})
    \]
    是一个有界函数。
  \end{block}

  \begin{block}{Tip: 与Lyapunov的联系}
    这个现象说明:
    如果一个函数$f$有下界并且为非增函数$\dot{f}\leq 0$,
    那么这个函数收敛到一个极限。
    \textbf{但是这个函数的导数不一定收敛到零}
  \end{block}
\end{frame}


\begin{frame}{Barbalat 引理}

  我们已知:若 $\lim_{t \to \infty} f(t) = 0$,则 $\lim_{t \to \infty} \intt{t}{f(\tau)} = C$(积分收敛,$C$ 为常数)。
  
  \vspace{0.5cm}
  \textbf{反过来成立吗?}  
  即:若 $\intt{\infty}{f(\tau)}$ 收敛(积分趋于常数),能否推出 $\lim_{t \to \infty} f(t) = 0$?

  \vspace{0.8cm}
  \begin{block}{反例:积分收敛,但函数不趋于零}
    回顾之前的三类函数(以矩形脉冲为例):
    \[ f_n(t) = n \cdot \chi_{[0, 1/n^2]}(t) \]
    ($\chi$ 为指示函数,$t \in [0,1/n^2]$ 时 $f_n(t)=n$,否则为0)
    
    - 积分特性:$\int_0^\infty f_n(t)dt = n \cdot \frac{1}{n^2} = \frac{1}{n} \to 0$(积分收敛);
    - 函数特性:$\max_{t \in \R} f_n(t) = n \to \infty$(函数不趋于零,反而发散)。
  \end{block}
\end{frame}

% ---------------------- 第3页:Barbalat 引理 —— 准确陈述 ----------------------
\begin{frame}{Barbalat 引理:条件与结论}
  Barbalat 引理是解决“积分收敛 $\Rightarrow$ 函数收敛”的关键工具,核心是通过**额外条件**弥补上述矛盾。

  \vspace{0.5cm}
  \begin{theorem}[经典 Barbalat 引理]
    设函数 $f(t): [0, \infty) \to \R$ 满足以下两个条件:
    \begin{enumerate}
      \item $f(t)$ 在 $[0, \infty)$ 上\textbf{连续};
      \item 无穷积分 $\intlim{0}{f(\tau)}$ \textbf{收敛}(即 $\lim_{t \to \infty} \intt{t}{f(\tau)} = C$,$C$ 为常数);
      \item $f'(t)$ 在 $[0, \infty)$ 上\textbf{一致连续}(或 $f(t)$ 本身一致连续)。
    \end{enumerate}
    则必有:$\boxed{\lim_{t \to \infty} f(t) = 0}$。
  \end{theorem}

  \vspace{0.3cm}
  \begin{remark}
    - 条件3是核心:若缺少“一致连续”,则可能出现“积分收敛但函数不趋于零”(如前页反例);
    - 常见简化版:若 $f(t)$ 一致连续且 $\intlim{0}{f(\tau)}$ 收敛,则 $\lim_{t \to \infty} f(t) = 0$(更易验证)。
  \end{remark}
\end{frame}

% ---------------------- 第4页:条件分析 —— 为什么需要“一致连续”? ----------------------
\begin{frame}{条件的必要性:反例验证}
  为说明“一致连续”不可缺少,我们对比两类函数(结合之前的Python绘图结果):

  \begin{columns}[T] % 分栏布局
    \column{0.5\textwidth}
    \textbf{反例1:矩形脉冲函数}
    \[ f_n(t) = n \cdot \chi_{[0, 1/n^2]}(t) \]
    \begin{itemize}
      \item 积分:$\int_0^\infty f_n(t)dt = 1/n \to 0$(收敛);
      \item 连续性:$f_n(t)$ 在 $t=1/n^2$ 处间断(非一致连续);
      \item 结论:$f_n(t)$ 不趋于零(最大值 $n \to \infty$)。
    \end{itemize}
    \vspace{0.3cm}
    % \includegraphics[width=0.9\textwidth]{rectangle_pulse.png} % 插入之前Python画的脉冲图

    \column{0.5\textwidth}
    \textbf{反例2:高频振荡函数}
    \[ f_n(t) = \sqrt{n} \cdot \sin(nt) \]
    \begin{itemize}
      \item 积分:$\int_0^\infty f_n(t)dt = \frac{1 - \cos(nt)}{\sqrt{n}} \to 0$(收敛);
      \item 连续性:$f_n(t)$ 连续,但 $f_n'(t) = n^{3/2}\cos(nt)$ 无界(非一致连续);
      \item 结论:$f_n(t)$ 振幅 $\sqrt{n} \to \infty$(不趋于零)。
    \end{itemize}
    \vspace{0.3cm}
    % \includegraphics[width=0.9\textwidth]{oscillating_function.png} % 插入振荡函数图
  \end{columns}
\end{frame}

% ---------------------- 第5页:证明思路 —— 简化版反证法 ----------------------
\begin{frame}{证明思路:反证法(简化版)}
  假设结论不成立,即 $\lim_{t \to \infty} f(t) \neq 0$,推出矛盾。

  \begin{enumerate}
    \item \textbf{反设前提}:存在 $\varepsilon_0 > 0$ 和序列 $t_k \to \infty$($k \to \infty$),使得 $\abs{f(t_k)} \geq \varepsilon_0$。
    
    \item \textbf{利用一致连续性}:因 $f'(t)$ 一致连续,存在 $\delta > 0$,对任意 $t \in [t_k - \delta, t_k + \delta]$,有:
    \[ \abs{f'(t)} \leq \frac{\varepsilon_0}{2\delta} \]
    由拉格朗日中值定理:$\abs{f(t) - f(t_k)} \leq \abs{f'(\xi)} \cdot \delta \leq \frac{\varepsilon_0}{2}$,故 $\abs{f(t)} \geq \frac{\varepsilon_0}{2}$。

    \item \textbf{积分矛盾}:对区间 $[t_k - \delta, t_k + \delta]$,积分满足:
    \[ \abs{\int_{t_k - \delta}^{t_k + \delta} f(t)dt} \geq \frac{\varepsilon_0}{2} \cdot 2\delta = \varepsilon_0 \delta > 0 \]
    但 $\intlim{0}{f(t)}$ 收敛,当 $k \to \infty$ 时,任意区间的积分应趋于0,矛盾。

    \item \textbf{结论}:反设不成立,故 $\lim_{t \to \infty} f(t) = 0$。
  \end{enumerate}
\end{frame}

% ---------------------- 第6页:核心应用 —— 控制理论中的稳定性分析 ----------------------
\begin{frame}{Barbalat 引理的核心应用:控制理论}
  Barbalat 引理是**Lyapunov稳定性分析**的重要补充,尤其用于证明“跟踪误差收敛”。

  \vspace{0.5cm}
  \begin{example}[自适应控制中的误差收敛]
    考虑跟踪系统:设跟踪误差为 $e(t)$,设计Lyapunov函数 $V(t)$ 满足:
    \[ \dot{V}(t) \leq -k e^2(t) \quad (k > 0 \text{ 为常数}) \]
    需证明 $\lim_{t \to \infty} e(t) = 0$,步骤如下:
    \begin{enumerate}
      \item 积分 $\dot{V}(t)$:$\int_0^t \dot{V}(\tau)d\tau = V(t) - V(0) \leq -k \int_0^t e^2(\tau)d\tau$;
      \item 因 $V(t) \geq 0$(Lyapunov函数性质),故 $\int_0^\infty e^2(\tau)d\tau \leq \frac{V(0)}{k}$(积分收敛);
      \item 验证一致连续:若系统输入有界,则 $\dot{e}(t)$ 有界 $\Rightarrow e(t)$ 一致连续;
      \item 应用Barbalat引理:$\int_0^\infty e^2(\tau)d\tau$ 收敛 + $e^2(t)$ 一致连续 $\Rightarrow \lim_{t \to \infty} e^2(t) = 0 \Rightarrow \lim_{t \to \infty} e(t) = 0$。
    \end{enumerate}
  \end{example}

  \vspace{0.3cm}
  \begin{block}{其他应用}
    - 信号处理:判断滤波后信号的收敛性;
    - 微分方程:证明解的渐近行为(如扰动系统的稳态误差)。
  \end{block}
\end{frame}

% ---------------------- 第7页:总结 ----------------------
\begin{frame}{总结:Barbalat 引理的核心价值}
  \begin{center}
    \begin{tabular}{|c|c|}
      \hline
      \textbf{核心问题} & 积分收敛 $\nRightarrow$ 函数收敛(需额外条件) \\
      \hline
      \textbf{Barbalat条件} & 1. 函数连续;2. 积分收敛;3. 导数(或函数)一致连续 \\
      \hline
      \textbf{关键结论} & 满足条件 $\Rightarrow \lim_{t \to \infty} f(t) = 0$ \\
      \hline
      \textbf{核心应用} & 控制理论(Lyapunov稳定性+误差收敛) \\
      \hline
    \end{tabular}
  \end{center}

  \vspace{1cm}
  \centering
  \textit{Barbalat引理的本质:用“一致连续性”约束函数的“波动幅度”,避免积分收敛但函数发散的情况。}
\end{frame}

\begin{frame}{Barbalat's lemma}
    
\end{frame}

\begin{frame}{核心思想:积分与函数的“分离性”}
  \begin{block}{关键区别}
    积分的“累积效应” ≠ 函数的“点态取值”:
    \begin{itemize}
      \item 积分趋于零:只需控制“函数高度 × 非零区间宽度”的乘积趋于零;
      \item 函数趋于零:需函数在“所有点”(或绝大多数点)的取值均趋于零。
    \end{itemize}
  \end{block}

  \vspace{0.5cm}
  \begin{center}
    \textbf{核心目标:构造满足“积分→0,但函数不→0”的函数}
  \end{center}
\end{frame}

% ---------------------- 第4页:例1 矩形脉冲函数 ----------------------
\section{场景1:固定区间上的函数序列}
\begin{frame}{例1:矩形脉冲函数(固定区间$[0,1]$)}
  \begin{columns}
    % 左列:函数定义与直观
    \column{0.45\textwidth}
    \begin{block}{1. 函数定义}
      对正整数 $n \to \infty$,定义:
      \[
      f_n(x) = 
      \begin{cases} 
      n & ,\ x \in \left[0, \frac{1}{n^2}\right] \\
      0 & ,\ x \in \left(\frac{1}{n^2}, 1\right]
      \end{cases}
      \]
      直观:“窄脉冲”——宽度$\frac{1}{n^2} \to 0$,高度$n \to \infty$。
    \end{block}

    % 右列:积分计算与趋势
    \column{0.55\textwidth}
    \begin{block}{2. 积分趋于零的证明}
      计算$[0,1]$上的定积分:
      \[
      \int_0^1 f_n(x) dx = \int_0^{\frac{1}{n^2}} n \, dx = n \times \frac{1}{n^2} = \frac{1}{n}
      \]
      当 $n \to \infty$ 时,$\frac{1}{n} \to 0$,故积分趋于零。
    \end{block}

    \begin{block}{3. 函数不趋于零的原因}
      取随$n$变化的点 $x_n = \frac{1}{2n^2}$,则:
      \[
      f_n(x_n) = n \to \infty \quad (n \to \infty)
      \]
      函数在“移动的窄区间”上取值无界,故不趋于零。
    \end{block}
  \end{columns}
\end{frame}

% ---------------------- 第5页:例2 高频震荡函数 ----------------------
\begin{frame}{例2:高频震荡函数(固定区间$[0,1]$)}
  \begin{columns}
    % 左列:函数定义与直观
    \column{0.45\textwidth}
    \begin{block}{1. 函数定义}
      对正整数 $n \to \infty$,定义:
      \[
      f_n(x) = \sqrt{n} \cdot \sin(nx) \quad (x \in [0,1])
      \]
      直观:高频震荡的正弦函数——振幅$\sqrt{n} \to \infty$,频率$n \to \infty$。
    \end{block}

    % 右列:积分计算与趋势
    \column{0.55\textwidth}
    \begin{block}{2. 积分趋于零的证明}
      计算定积分:
      \[
      \int_0^1 \sqrt{n}\sin(nx)dx = \sqrt{n} \cdot \left. \frac{-\cos(nx)}{n} \right|_0^1 = \frac{1 - \cos(n)}{\sqrt{n}}
      \]
      因 $|\cos(n)| \leq 1$,故 $|1 - \cos(n)| \leq 2$,进而:
      \[
      \left| \frac{1 - \cos(n)}{\sqrt{n}} \right| \leq \frac{2}{\sqrt{n}} \to 0 \quad (n \to \infty)
      \]
    \end{block}

    \begin{block}{3. 函数不趋于零的原因}
      取 $x_n = \frac{\pi}{2n}$,则 $\sin(nx_n) = 1$,故:
      \[
      f_n(x_n) = \sqrt{n} \to \infty \quad (n \to \infty)
      \]
      振幅无界,函数不趋于零。
    \end{block}
  \end{columns}
\end{frame}

% ---------------------- 第6页:例3 无穷区间函数 ----------------------
\section{场景2:无穷区间上的函数}
\begin{frame}{例3:傅里叶积分型函数 $f(x) = \sin(x^2)$(区间$[1,+\infty)$)}
  \begin{block}{1. 函数定义与直观}
    定义:$f(x) = \sin(x^2)$($x \in [1,+\infty)$),特点是“频率递增的震荡函数”:
    - $x^2$ 增速随$x$增大而加快,导致$\sin(x^2)$震荡越来越频繁;
    - 振幅始终为1(有界但不趋于零)。
  \end{block}

  \begin{block}{2. 局部积分趋于零($A \to +\infty$)}
    计算局部积分 $\int_A^{A+1} \sin(x^2)dx$,作变量代换 $t = x^2$($dx = \frac{dt}{2\sqrt{t}}$):
    \[
    \int_A^{A+1} \sin(x^2)dx = \int_{A^2}^{(A+1)^2} \frac{\sin t}{2\sqrt{t}} dt
    \]
    由积分中值定理,存在 $\xi \in [A^2, (A+1)^2]$ 使:
    \[
    \text{积分} = \frac{\sin \xi}{2\sqrt{\xi}} \cdot (2A+1)
    \]
    因 $\sqrt{\xi} \geq A$,故 $\frac{2A+1}{2\sqrt{\xi}} \leq \frac{2A+1}{2A} \to 1$,且 $|\sin \xi| \leq 1$,因此:
    \[
    |\text{积分}| \leq \frac{2A+1}{2A} \cdot \frac{1}{\sqrt{\xi}} \to 0
    \]
  \end{block}

  \begin{block}{3. 函数不趋于零的原因}
    取点列 $x_k = \sqrt{\frac{\pi}{2} + 2k\pi}$($k \to +\infty$),则:
    \[
    f(x_k) = \sin\left(\frac{\pi}{2} + 2k\pi\right) = 1 \not\to 0
    \]
    函数始终在$[-1,1]$震荡,无极限。
  \end{block}
\end{frame}

% ---------------------- 第7页:核心总结 ----------------------
\section{核心总结}
\begin{frame}{核心总结}
  \begin{block}{构造“积分→0但函数不→0”的两种核心机制}
    \begin{enumerate}
      \item \textbf{脉冲机制}:
        - 函数仅在“宽度→0”的区间上非零,高度→∞,但“高度×宽度→0”(如矩形脉冲);
        - 关键:用“窄区间”抵消“高取值”的累积效应。

      \item \textbf{震荡抵消机制}:
        - 函数高频震荡,正负面积相互抵消,积分→0,但振幅不→0(如$\sqrt{n}\sin(nx)$、$\sin(x^2)$);
        - 关键:用“高频震荡”实现面积抵消。
    \end{enumerate}
  \end{block}

  \vspace{0.5cm}
  \begin{alertblock}{重要结论}
    \centering
    \textbf{“积分趋于零”不能推出“函数趋于零”} \\
    积分收敛反映“累积效应→0”,而非“点态取值→0”。
  \end{alertblock}
\end{frame}
