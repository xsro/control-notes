\subsection{自适应估计}


\begin{frame}
    \frametitle{PE条件与参数估计}
    自适应控制与系统辨识中最基本的问题是,从标量输出 \( y \in \mathbb{R} \) 估计参数向量 \( \theta \in \mathbb{R}^n \),它们满足线性回归关系
    \[
    y(t) = \phi(t)^\top \theta.
    \]

    若要得唯一解,则至少有 \( n \) 个时刻方阵 \( [\phi(t_1), \dots, \phi(t_n)] \) 满秩,这是所谓的可辨识性。在控制问题中我们倾向于持续地在线估计,这样既可保证鲁棒性,又可以在线跟踪缓慢的参数变化。这就要把可辨识性,即 \( [\phi(t_1), \dots, \phi(t_n)] \) 满秩,大体上“一直保持下去”,以上即持续激励的基本出发点。对于以上线性回归问题,最基本的梯度下降法,可得如下线性时变误差系统
    \[
    \dot{\hat{\theta}}= -\gamma \phi(t) y(t)
    \]\[
    \dot{\tilde{\theta}} = -\gamma \phi(t) \phi(t)^\top \tilde{\theta}.
    \]
    其中$\tilde{\theta}=\hat{\theta}-\theta$.
\end{frame}

\begin{frame}
    \frametitle{PE条件的理解 1}

    \begin{definition}[持续激励]
        \parencite{NARENDRA1987}
        设 $\omega: [t_0, \infty) \to \mathbb{R}^p$ 是时变参数,其初始条件定义为 $\omega_0 = \omega(t_0)$,则关于时间的参数化函数 $y(t,\omega): [t_0,\infty) \times \mathbb{R}^p \to \mathbb{R}^m$ 满足:
        若存在 $T > 0$ 和 $\alpha > 0$,使得
    \[
        \int_{t}^{t+T} y(\tau,\omega) y^\top(\tau,\omega) d\tau \succeq \alpha I
    \]
    对\textcolor{teal}{所有 $t \geq t_0$} 和 \tB{$\omega_0 \in \mathbb{R}^p$} 成立,则称 $y(t,\omega) \in \text{PE}$。
    \end{definition}
    \begin{itemize}
    \item 从几何的角度来看,持续激励是说,在每个时间段 \( [t, t+T] \) 内,\( \phi(t) \) 旋转过方向所张成的空间要能够覆盖整个 \( \mathbb{R}^n \) 空间。显然对于任意时刻 \( \text{rank}\{\phi(t) \phi(t)^\top\} = 1 \),只有在每个时间段 \( [t, t+T] \) 内,\( \phi(t) \) 旋转过 \( n \) 个线性无关的“方向”,才能让以上持续激励条件成立。
\end{itemize}
\end{frame}

\begin{frame}
    \frametitle{PE条件的理解 2}
    \begin{itemize}
        \item 从平均理论的角度来说,持续激励是让(时变的)系统矩阵始终为正定。误差系统的平均动力学“看上去”是(ODE教材里有严格的平均分析)
        \[
        \dot{\bar{\tilde{\theta}}} = -\gamma \bar{A}(t) \bar{\tilde{\theta}}, \quad \bar{A} := \frac{1}{T} \int_{t}^{t+T} \phi(\tau)\phi(\tau)^\top d\tau
        \]
        其中 \( \bar{\tilde{\theta}} \) 是 \( \tilde{\theta} \) 的平均变量。平均系统矩阵 \( \bar{A}(t) \) 在持续激励条件下为正定,状态将收敛至平衡点。

        \item 从频域角度来看,持续激励条件要求系统输入的频谱丰富程度大于等于 \( n \)。最简单的情况就是回归矩阵 \( \phi(t) \) 是由单入单出线性时不变系统产生,那么该LTI系统输入含有不少于 \( n \) 个频谱时,即输入 \( u \) 可以分解为
        \[
        u = \sum_{i=1}^{m} \sin(\omega_i t + \psi_i)
        \]
        其中 \( m \geq n/2 \)(每个正弦信号提供2个频谱),那么回归矩阵 \( \phi(t) \) 是持续激励的。这也是充分必要条件。
    \end{itemize}
\end{frame}

\begin{frame}
    \frametitle{如何保证满足PE}

    工程中如何保证信号满足PE条件很困难!
    常常需要给一个随机信号。

    于是出现了一些改进的PE条件表述,如PE*\parencite{Jenkins2018},初始激励/充分激励条件(initial exciting, IE/sufficiently exciting, SE)。

    \begin{enumerate}
    \item \underline{SE/IE条件}:$\exists T>0 ,s.t. \int_{0}^{T} y(\tau,\omega) y^\top(\tau,\omega) d\tau \succeq \alpha I$
    \item     \underline{弱持续激励}($\text{PE}^*(\omega, \Omega)$):若\tB{存在紧集 $\Omega \subset \mathbb{R}^p$}、$T(\Omega) > 0$ 及 $\alpha(\Omega)$,使得
    \[
        \int_{t}^{t+T} y(\tau,\omega) y^\top(\tau,\omega) d\tau \succeq \alpha I
    \]
    对所有 $\omega_0 \in \Omega$ 和 $t \geq t_0$ 成立,则称 $y(t,\omega) \in \text{PE}^*(\omega, \Omega)$。
    \end{enumerate}
\end{frame}