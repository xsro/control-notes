\subsection{自适应估计}


\begin{frame}
    自适应控制与系统辨识中最基本的问题是,从标量输出 \( y \in \mathbb{R} \) 估计参数向量 \( \theta \in \mathbb{R}^n \),它们满足线性回归关系
\[
y(t) = \phi(t)^\top \theta.
\]

若要得唯一解,则至少有 \( n \) 个时刻方阵 \( [\phi(t_1), \dots, \phi(t_n)] \) 满秩,这是所谓的可辨识性。在控制问题中我们倾向于持续地在线估计,这样既可保证鲁棒性,又可以在线跟踪缓慢的参数变化。这就要把可辨识性,即 \( [\phi(t_1), \dots, \phi(t_n)] \) 满秩,大体上“一直保持下去”,以上即持续激励的基本出发点。对于以上线性回归问题,最基本的梯度下降法,可得如下线性时变误差系统
\[
\dot{\hat{\theta}}= -\gamma \phi(t) y(t)
\]\[
\dot{\tilde{\theta}} = -\gamma \phi(t) \phi(t)^\top \tilde{\theta}.
\]
\end{frame}

\begin{frame}
    \begin{itemize}
    \item 从动力系统稳定性角度来看,以上系统原点指数稳定 \( \Leftrightarrow \phi(t) \) 是持续激励的,即存在常数 \( T, k \) 使得在任意时刻 \( t \) 都有
    \[
    \int_{t}^{t+T} \phi(\tau) \phi(\tau)^\top d\tau \geq k I_{n \times n}.
    \]
    \item 从几何的角度来看,持续激励是说,在每个时间段 \( [t, t+T] \) 内,\( \phi(t) \) 旋转过方向所张成的空间要能够覆盖整个 \( \mathbb{R}^n \) 空间。显然对于任意时刻 \( \text{rank}\{\phi(t) \phi(t)^\top\} = 1 \),只有在每个时间段 \( [t, t+T] \) 内,\( \phi(t) \) 旋转过 \( n \) 个线性无关的“方向”,才能让以上持续激励条件成立。
\end{itemize}
\end{frame}

\begin{frame}
    \begin{itemize}
    \item 从平均理论的角度来说,持续激励是让(时变的)系统矩阵始终为正定。误差系统的平均动力学“看上去”是(ODE教材里有严格的平均分析)
    \[
    \dot{\bar{\tilde{\theta}}} = -\gamma \bar{A}(t) \bar{\tilde{\theta}}, \quad \bar{A} := \frac{1}{T} \int_{t}^{t+T} \phi(\tau)\phi(\tau)^\top d\tau
    \]
    其中 \( \bar{\tilde{\theta}} \) 是 \( \tilde{\theta} \) 的平均变量。平均系统矩阵 \( \bar{A}(t) \) 在持续激励条件下为正定,状态将收敛至平衡点。

    \item 从频域角度来看,持续激励条件要求系统输入的频谱丰富程度大于等于 \( n \)。最简单的情况就是回归矩阵 \( \phi(t) \) 是由单入单出线性时不变系统产生,那么该LTI系统输入含有不少于 \( n \) 个频谱时,即输入 \( u \) 可以分解为
    \[
    u = \sum_{i=1}^{m} \sin(\omega_i t + \psi_i)
    \]
    其中 \( m \geq n/2 \)(每个正弦信号提供2个频谱),那么回归矩阵 \( \phi(t) \) 是持续激励的。这也是充分必要条件。
\end{itemize}
\end{frame}