\section{引言}
\subsection{什么是自适应控制}

\begin{frame}{韦氏词典对“适应”与“适应性”的定义}
	\begin{itemize}
		\item According to the Webster’s dictionary, to \textbf{adapt} means:
		\begin{itemize}
			\item to adjust oneself to particular conditions
			\item to bring oneself in harmony with a particular environment
			\item to bring one’s acts, behaviour in harmony with a particular environment
		\end{itemize}
		\item 【中文翻译】根据韦氏词典,“adapt”(适应)的含义是:
		\begin{itemize}
			\item 使自己适应特定条件
			\item 使自己与特定环境相协调
			\item 使自己的行为举止与特定环境相协调
		\end{itemize}
	\end{itemize}
	\begin{itemize}
		\item According to the Webster’s dictionary, \textbf{adaptation} means:
		\begin{itemize}
			\item adjustment to environmental conditions
			\item alteration or change in form or structure to better fit the environment
		\end{itemize}
		\item 【中文翻译】根据韦氏词典,“adaptation”(适应性)的含义是:
		\begin{itemize}
			\item 对环境条件的调整
			\item 为更好地适应环境而在形式或结构上的改变或调整
		\end{itemize}
	\end{itemize}
\end{frame}



\begin{frame}{自适应控制器的定义与理论核心}
	\begin{itemize}
		\item An adaptive controller is a \textbf{fixed-structure} controller with \textbf{adjustable parameters} and a mechanism for \textbf{automatically} adjusting those parameters
		\item 【翻译】自适应控制器是一种\textbf{固定结构}的控制器,具有\textbf{可调参数}以及自动调整这些参数的机制
		\item In this sense, an adaptive controller is one way of dealing with \textbf{parametric uncertainty}
		\item 【翻译】从这个意义上说,自适应控制器是应对\textbf{参数不确定性}的一种方法
		\item Adaptive control theory essentially deals with finding parameter adjustment algorithms that guarantee global \textbf{stability and convergence}
		\item 【翻译】自适应控制理论本质上是研究寻找能保证\textbf{全局稳定性和收敛性}的参数调整算法
	\end{itemize}
\end{frame}


\begin{frame}
	\frametitle{什么是自适应控制}
	Adaptive Control covers a set of techniques which provide a systematic approach for
	automatic adjustment of controllers in real time, in order to achieve or to maintain
	a desired level of control system performance when \textbf{the parameters of the plant
	dynamic model are unknown and/or change in time.}
	
	[1] Landau I D, Lozano R, M’Saad M, et al. Adaptive Control: Algorithms, Analysis and Applications[M]. London: Springer London, 2011.
	
	\vspace{1cm}
	
	\textbf{自适应控制}:通过对于未知模型参数进行\textbf{在线估计},然后将参数估计值带入并更新控制器的设计。
	
	【CAA云讲座预告】北京航空航天大学教授王薇:数据驱动自适应控制理论及应用
	
\end{frame}

\begin{frame}
	\frametitle{自适应与自整定}
	Self-tuning 自整定
	\begin{itemize}
		\item Continuous updating of controller parameters 控制器参数的持续更新
		\item Used for truly time-varying plants 用于真正时变的被控对象
	\end{itemize}
	
	Auto-tuning
	\begin{itemize}
		\item Once controller parameters near convergence, adaptation is stopped 一旦控制器参数接近收敛,自适应即停止
		\item Used for time invariant or very slowly varying processes 用于时不变或变化极慢的过程
		\item Used for periodic, usually on-demand tuning 用于周期性的、通常为按需的整定
	\end{itemize}
\end{frame}


\begin{frame}
	\frametitle{自适应与增益调度}
	
	\begin{columns}[t]
		\column{0.48\textwidth}
		\begin{figure}
			\includegraphics[height=6cm]{figure/gain_schedualing}
		\end{figure}
		
		\column{0.48\textwidth}
		自适应控制是 “以变应变的智能调节”,增益调度控制是 “按预设规则的被动切换”。前者适合未知时变场景,后者适合已知工况的规律变化场景。
		
		\begin {itemize}\item 时变动态系统的控制\item 若动态特性随运行工况以已知、可预测的方式变化,采用 \textbf {增益调度}\item 若使用固定控制器无法在 \textbf {鲁棒性} 和 \textbf {性能} 之间达成满意的折中,当且仅当此时,才应使用自适应控制。\end {itemize}
		
		\textbf {使用满足技术要求的最简单技术}
	\end{columns}
	
\end{frame}


\subsection{为什么使用自适应控制}


\begin{frame}{传统控制的局限性:为什么需要新方案?}
	\begin{block}{工业系统的普遍挑战:不确定性}
		实际工程中,系统往往存在难以预先建模的\textbf{内外部不确定性},导致传统固定参数控制器失效:
		\begin{itemize}
			\item \textbf{内部不确定性}:
			\begin{itemize}
				\item 参数时变(如化工反应釜催化剂活性衰减、电机老化导致电阻变化)
				\item 未建模动态(如高层建筑风振、柔性机械臂弹性形变)
			\end{itemize}
			\item \textbf{外部不确定性}:
			\begin{itemize}
				\item 环境干扰(如无人机遭遇阵风、电网电压波动)
				\item 任务切换(如机器人负载突变、飞行器飞行模式转换)
			\end{itemize}
		\end{itemize}
	\end{block}
	
	\begin{exampleblock}{传统控制的“困境”}
		固定参数控制器仅针对“理想工况”设计,当系统特性偏离预设模型时,会出现:
		\begin{center}
			控制精度下降 $\to$ 系统稳定性恶化 $\to$ 甚至引发安全事故
		\end{center}
	\end{exampleblock}
\end{frame}




\begin{frame}{自适应控制的核心价值:解决传统方法无法应对的场景}
	\begin{columns}[t]
		\column{0.5\textwidth}
		\begin{block}{传统控制的“被动适应”局限}
			\begin{itemize}
				\item 增益调度控制:仅能应对\textbf{已知规律的工况变化}(如预设飞机马赫数-参数表),无法处理突发扰动
				\item 鲁棒控制:通过“保守设计”保证稳定性,但牺牲了\textbf{动态性能}(如响应速度变慢)
				\item 最优控制:依赖精确模型,模型失配时会导致\textbf{性能剧烈下降}
			\end{itemize}
		\end{block}
		
		\column{0.5\textwidth}
		\begin{block}{自适应控制的“主动调节”优势}
			针对不确定性,自适应控制通过“在线辨识+实时调参”实现:
			\begin{itemize}
				\item 无需预知参数变化规律,自动跟踪\textbf{未知时变特性}
				\item 在保证稳定性的同时,维持\textbf{高性能指标}(如最小跟踪误差)
				\item 适用于“模型难以精确建立”的复杂系统(如生物反应器、自动驾驶)
			\end{itemize}
		\end{block}
	\end{columns}
	
	\vspace{1em}
	\begin{center}
		\textbf{核心差异:从“以不变应万变”到“以变应变”}
	\end{center}
\end{frame}




\begin{frame}{实际应用中的迫切需求:自适应控制的不可替代性}
	\begin{block}{典型场景:为什么必须用自适应控制?}
		\begin{enumerate}
			\item \textbf{航空航天领域}:
			飞行器在宽空域(高度0-30km)、宽速域(马赫数0.3-5)飞行时,气动参数变化可达\textbf{10倍以上},传统控制器无法全覆盖,需自适应控制保证姿态稳定(如NASA X-43高超声速飞行器)。
			
			\item \textbf{工业制造领域}:
			13万吨级造纸机换产时,纸张定量、湿度等参数突变,自适应控制可将调整时间从传统方法的2小时缩短至\textbf{15分钟},降低废品率30%(某造纸厂实际案例)。
			
			\item \textbf{机器人与智能装备}:
			服务机器人抓取不同重量物体(0.1-5kg)时,动力学参数剧变,自适应控制可实现无超调的力/位置跟踪,避免传统控制的震荡问题。
		\end{enumerate}
	\end{block}
	
	\begin{alertblock}{总结:自适应控制的核心目标}
		在“模型不精确、参数时变、干扰未知”的场景下,实现:
		\begin{center}
			稳定运行 $\to$ 性能优化 $\to$ 降低人工干预 $\to$ 扩展系统适用范围
		\end{center}
	\end{alertblock}
\end{frame}


\begin{frame}
	\frametitle{没有自适应控制的时候}
	\begin{figure}
		\includegraphics[height=6cm]{figure/PID_pure}
	\end{figure}
\end{frame}


\begin{frame}
	\frametitle{有了自适应控制的时候}
	\begin{figure}
		\includegraphics[height=6cm]{figure/PID_adaptive}
	\end{figure}
\end{frame}

% \subsection{自适应控制的发展历史}

% \begin{frame}{自适应控制发展历程(一):理论奠基期(1950s-1960)}
% 	\begin{itemize}
% 		\item Mid 1950's: Flight control systems (eventually solved by gain scheduling)
% 		\item 1957: Bellman develops dynamic programming
% 		\item 1958: Kalman develops the self-optimizing controller \\
% 		``which adjusts itself automatically to control an arbitrary dynamic process''
% 		\item 1960: Feldbaum develops the dual controller in which the control action serves a dual purpose as it is ``directing as well as investigating''
% 	\end{itemize}
% \end{frame}

% \begin{frame}{自适应控制发展历程(二):技术成型与深化期(1960s-1990s)}
% 	\begin{itemize}
% 		\item Mid 60's-early 70's: Model reference adaptive systems
% 		\item Late 60's-early 70's: System identification approach with recursive least-squares
% 		\item Early 1980's: Convergence and stability analysis
% 		\item Mid 1980's: Robustness analysis
% 		\item 1990's: Multimodel adaptive control
% 		\item 1990's: Iterative control
% 	\end{itemize}
% \end{frame}