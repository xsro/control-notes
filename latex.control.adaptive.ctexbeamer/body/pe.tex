\subsection{PE 条件}

\begin{frame}
    \frametitle{PE条件:一个充分条件}

    

    \begin{lemma}
        Lemma 2.4 \cite{Chen2015} Sec 2.3 p 29.
        考虑一个连续可微的函数$g:[0,\infty)\mapsto \mathbb{R}^n$ 和一个有界分段连续的函数$f:[0,\infty)\mapsto \mathbb{R}^n$满足
        \[\lim_{t\to\infty} g^T(t)f(t)=0\]
        当满足以下两个条件的时候:
        \begin{enumerate}
            \item $\lim_{t\to\infty} \dot{g}(t)=0$
            \item $f(t)$ is PE.
        \end{enumerate}
        有 $\lim_{t\to\infty}g(t)=0$.
    \end{lemma}
\end{frame}

\begin{frame}
    \frametitle{PE条件仿真对比}
    \begin{enumerate}
        \item 持续激励信号:$w(t)=\sin(t)$.
        \item 非持续激励信号:当 $t< 2\pi$ 时,$w(t)=\sin(t)$;当$t\geq 5$时,$w(t)=0$.
    \end{enumerate}

    \begin{figure}
        \begin{subfigure}[b]{0.45\textwidth}
            \includegraphics[width=\linewidth]{python/out/simple-pe.pdf}
            \caption{持续激励信号下}
        \end{subfigure}
        \begin{subfigure}[b]{0.45\textwidth}
            \includegraphics[width=\linewidth]{python/out/simple-npe.pdf}
            \caption{非持续激励信号}
        \end{subfigure}
    \end{figure}

    \begin{block}{什么条件能保证系统稳定}
        \begin{enumerate}
            \item 猜想1:$w(t)$ 非零信号?
            \item 猜想2:$w(t)$ 非恒定值信号?
        \end{enumerate}
    \end{block}
\end{frame}

