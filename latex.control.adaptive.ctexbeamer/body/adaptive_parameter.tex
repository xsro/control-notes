\section{自适应控制}

\subsection{未知参数自适应控制}

\begin{frame}
    \frametitle{参数自适应}

    考虑一维镇定问题($x,w,u\in[0,\infty)\mapsto\mathbb{R}$):
    \begin{equation}
        \dot{x}=w(t) \mu +u
    \end{equation}
    其中$\mu\in\mathbb{R}$是未知恒定参数,设计控制器为:
    \begin{equation}
        u=-x-w(t)\hat{\mu},
        \quad
        \dot{\hat{\mu}}=-xw(t)
    \end{equation}

    于是闭环的自适应控制系统可以写作:
    \[
    \begin{cases}
    \dot{e} = -e + \tilde{\mu} w(t) \\
    \dot{\tilde{\mu}} = -e w(t)
    \end{cases}
    \]

    其中,\( e=x \) 是跟踪误差,\( \tilde{\mu}=\hat{\mu}-\mu \) 是参数误差,\( w(t) \) 是一个有界连续函数。
\end{frame}

\begin{frame}
考虑下方有界函数:
\[
\begin{split}
V &= e^2 + \tilde{\mu}^2 \\
\dot{V} &= 2e(-e + \tilde{\mu} w) + 2\tilde{\mu}(-ew(t)) = -2e^2 \leq 0
\end{split}
\]
所以 \( V(t) \leq V(0) \),因此 \( e \) 和 \( \tilde{\mu} \) 是有界的。

由于系统的$\dot{V}\leq 0$,无法直接保证渐进稳定。
由于动力学是非自治的,\textbf{不变集定理}不能用来推断 \( e \) 的收敛性。
因此,需要使用基于Barbalat's 引理的LaSalle-Yoshizawa定理。

通过$\ddot{V}$的有界性,检查 \( \dot{V} \) 的一致连续性。
\[
\ddot{V} = -4e(-e + \tilde{\mu} w(t))
\]

\( \ddot{V} \) 是有界的,因为根据假设 \( w \) 是有界的,且 \( e \) 和 \( \tilde{\mu} \) 已被证明是有界的 \( \rightsquigarrow \dot{V} \) 是一致连续的。

应用 Barbalat 引理:当 \( \dot{V} = 0 \) 时,\( t \to \infty \) 时 \( e \to 0 \)。

\textbf{重要提示}:尽管 \( e \to 0 \),但整个系统不是渐近稳定(a.s.)的,因为仅证明了 \( \tilde{\mu} \) 是有界的。
\end{frame}


\begin{frame}
    \frametitle{什么时候观测误差也渐进收敛?}

    通过分析已知$e\to 0$, $\dot{e}\to0$(Barbalat's lemma).

    \[
    \begin{cases}
    \dot{e} = -e + \tilde{\mu} w(t) \\
    \dot{\tilde{\mu}} = -e w(t)
    \end{cases}
    \]
    第一行:$\tilde{\mu} w(t)=\dot{e}+e\to 0$.
    第二行:$\dot{\tilde{\mu}} = -e w(t)\to 0$.

    \begin{block}{$w(t)$需要满足什么条件能够得到$\tilde{\mu}(t)\to 0$}
        \footnote{示例来源于\cite[Example 2.7]{Chen2015}}

    如果$w(t)$是一个非零的常数,可得$\lim_{t\to\infty}\tilde{\mu}(t)=0$.
    
    如果$w(t)=[\sin(\omega t),\sin(\omega t)]^T$的时候,那么显然存在$\tilde{\mu}(t)=[1 -1]^T$的解,不满足$\lim_{t\to\infty}\tilde{\mu}(t)=0$.
        
    \end{block}
\end{frame}